\section{Konsistenzprüfungen}

Es gibt verschiedene Modus als Vorschlag
 * nach jeder Regelanwendung ('''nicht sinnvoll''': es kann durch eine Regelanwendung nicht gegen beliebige Konsistenzregeln verstoßen werden)
 * nach kompletter Ableitung ('''fraglich''', es sollte zwar gehen, aber ob es soviel schneller ist als nur die abhängigen ist unbekannt)
 * nur abhängige Konsistenzregeln ('''default''': sicher und nur nötigster Aufwand)
 * keine ('''schnell''', wenn bekannt ist, das die Ontologie konsistent ist braucht ja nicht jedesmal ihre Konsistenz wieder überpürft werden)
Modusname: '''!ConsistencyLevel''' := NONE | DEFAULT

== Einsatzort ==
Die Tabellen sollten hier auch wissen, welche (Konsistenz-)Regeln sie auslösen (können), der !ReasonProcessor entscheidet dann, welche !RuleActions er anlegt.

== Rückmeldung ==
Konsistenzregeln melden sich nur in einem Fehlerfall zurück. Die Rückmeldung erfolgt aus den Regeln selbst während ihrer Ausführung.

Hier gibt es die zwei Möglichkeiten:
 * nur eine Warnung auszugeben (log4j)
 * die Verarbeitung unterbrechen (Exception werfen)
 * abhängig von der Inkonsistenz reagieren
Modusname: '''!InconsitencyReaction''' := WARN | FAIL | PERCASE

== Implementierung ==
Von der Klasse Rule wird eine Subklasse !ConsistencyRule abgeleitet von der alle Konsistenzregeln abgeleitet sein sollen.