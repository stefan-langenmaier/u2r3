\chapter{Bewertung}
Diese Diplomarbeit enhält die Lösung vieler grundlegender Probleme, die bei der Anpassung eines Schlussfolgerers in ein wirkliches System (OWLAPI) entstehen. Die Anwendung ist zwar noch nicht fertig\footnote{Eine Themenbereich der sich mit Optimierung beschäftigt wird wahrscheinlich nie den Status fertig erreichen.} bzw. erfüllt alle Anforderungen für ein verkaufbares Produkt. Es sind aber Rezepte für alle Bereich aufgezeigt worden, wie diese umzusetzen sind.

Die Bewertung der Leistung des Schlussfolgerers kann dem Abschnitt \ref{abschnitt-vergleiche} entnommen werden.

In dieser Diplomarbeit fand nicht nur die Implementierung & Optimierung eines relationalen Schlussfolgerers statt. Es wurden Konzepte aus früheren Arbeiten überprüft, überarbeitet, erweitert und miteingebaut. Es wurde mit sich noch in der Entwicklung befinlichen Schnittstellen (OWLAPI) gearbeitet und dafür Verbesserungsvorschläge\footnote{Im Bugtracker wurde Korrekturen für die OWLAPIv3 eingebracht.} bei anderen Entwicklern gemacht. Es wurden Korrekturen für Spezifikationen (OWL2 RL) eingesendet. Es wurden eigene Ideen zur Optimierung formuliert\footnote{siehe Abschnitt \ref{abschnitt-weitere-optimierungen}} und umgesetzt. Es wurden verschiedene Quellen zum Testen des Schlussfolgerers herangezogen und auch eigene entwickelt. Der entwickelte Schlussfolgerer wurde mit unterschiedlichen Schlussfolgerungssystemen verglichen. Es wurde eine Erweiterung über den OWL2 RL Regelsatz hinaus implementiert.

Das sind alles Ergebnisse, die in dieser Diplomarbeit entstanden sind. Die zentralen Kerfragen waren aber schon zu Beginn:
\begin{enumerate}
  \item Lohnt sich der Umstieg auf ein RDBMS? Das kann mit einem ja beantwortet werden. In einem RDBMS sind alle Operationen möglich die in u2r2 möglich waren. Allerdings soart man sich dadurch Code, hat optimierte Alogrithmen, kann Fehler besser isolieren und hat die Freiheit dasdarunterliegende System zu wechseln.
  \item Lohnt sich der Speicher-Tradeoff gegenüber der Anfragezeit, d.h. kann man es verantworten alle möglichen Fakten abzuspeichern? Der Speicher-Tradeoff lohnt sich auf alle Fälle. Der Platz auf Festplatten ist im Gegensatz zu RAM fast unbegrenzt bzw wesentlich kostengünstiger. Dabei konnte man den Vergleichen zwischen den verschiedenen Reasoner entnehmen, das die meisten bei der VAST-Ontologie mehr RAM benötigten als u2r3 für das ABspeichern in der Datenbank\footnote{Der u2r3 hat dafür 180MB benötigt.}.
  \item Lohnt sich die Vorbereitungszeit gegenüber der Anfragezeit? Der Vorbereitungs-Tradeoff wird vom Anwedungsgebiet bestimmt. Wird eine Ontologie nur einmal geladen und es werden viele Anfgragen oder auch kleine Änderungen an der Ontologie vorgenommen kann sich die Arbeitsweise des u2r3 schnell auszahlen. Vorall bei großen Ontologien ist der relative Unterschied gering.
  \item Wie können Änderung effizient realisert werden? Das wurde durch eine Ableitungshistorie umgesetzt, die gezielt gewisse Ableitungen von Axiomen rückgängig machen kann.
\end{enumerate}
