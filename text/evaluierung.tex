\chapter{Evaluierung}
Die Evaluierung sollen einen Vergleich von u2r3 mit anderen Schlussfolgereren ermöglichen, um seine Leistung in Relations zu sehen. Es sollen die Vor- und Nachteile mit anderen Anwendungen gegenüber gestellt werden.

\section{Konkurrenz}
Die ``großen'' Schlussfolgerer im Moment sind FaCT++ \cite{Factpp}, Pellet \cite{Pellet}, HermiT \cite{HermiT}, RacerPro \cite{RacerPro} und OWLim \cite{OWLim}. Mit der Ausnahme von OWLim sind dabei alle Tableaux-basiert und für den DL- Bereich ausgelegt. Von der Heransgehenweise ist dabei OWLim der einzig Vergleichbare, da er ebenfalls auf einem direct-materialisation und zu mindest teilweise forward-chaining Ansatz basiert. Allerdings gibt es hier größere Unterschiede bzgl. der Anbindung nach außen (\mbox{SERQL}/\mbox{SPARQL}) und auch dem Sprachbereich in dem geschlussfolgert wird.

Eine Anbindung an die OWLAPI ist zwar bei allen außer OWLim vorhanden, jedoch hat mit der Ausnahme von HermiT keiner eine lauffähige Version bei der Erscheinung der OWLAPIv3 vorweisen können. HermiT ist ein jüngerer Reasoner der ziemlich rapide entwickelt werden.

Diese Punkte machen es alle recht schwer einen brauchbaren Vergleich zu u2r3 zu finden.

\section{Vergleiche}
\label{abschnitt-vergleiche}
Die Implementierung und Optimierung von u2r3 hat sich hauptsächlich mit dem Bereich des Schlussfolgerns und nicht mit der Beantwortung von Abfragen beschäftigt. Die Beantwortung von Anfragen ist sicher auch ein großer und komplexer Teil -- einer der Gründe warum dies nur rudimentär implementiert wurde -- aber durch den direct-materialisation Ansatz sollten alle Ergebnisse möglicher Anfragen schon zur Anfragezeit vorliegen und der begrenzende Faktor ist somit hauptsächlich die Datenbank und die Geschwindigkeit mit der die Daten zurückgeliefert werden können.


\begin{itemize}
  \item T-Box Klassifizierung: Dafür wurden die instrument ($\mathcal{AL}$) und diseases Ontologien ($\mathcal{AL}$) ausgewählt, da sie fast ausschließlich aus Konzept-Hierachien bestehen. Die instrument ist eine Ontologie mittlerer Größe, die diseases ist eine große Ontologie.
  \item A-Box Realisierung: Hier wurden die Ontologien financial ($\mathcal{ALCOIF}$) und VAST2009 ($\mathcal{ALCOIF}(D)$) sowie VAST2009-tiny ($\mathcal{ALCOIF}(D)$) ausgewählt. Die ersten beiden zeichnen sich durch eine große Anzahl an Individuen aus. VAST2009-tiny ist dabei eine modifizierte Version der VAST2009, die in ihrer Größ stark reduziert wurde. Die financial Ontologie enthält dabei nur objectProperties, die VAST2009 Ontologie enthält hauptsächlich dataProperties und muss daher mit Literalen arbeiten.
  \item OWL2 RL: walter ($\mathcal{ALEROIQ}+$), als ein Beispiel für die Ausdrucksmächtigkeit von OWL2 RL
  \item OWL2 RL + Extras: DomusAG-full ($\mathcal{ALCHOIF}(D)$) als ein Beispiel für die DatatypeRestriction Erweiterung
\end{itemize}

Alle Ontologien finden sich im Anhang und im Repository im Ordner \texttt{ontologien/}.

Der Ablauf der Tests ist in drei Teile untergliedert, das Laden der Ontologie, das Hinzufügen eines Axioms zur Ontologie und das Löschen eines Axioms aus dieser Ontologie. In jedem Teil wird eine Anfrage passend zur Ontologie gestellt. Es wird zweimal die Zeit gemessen. Einmal für das bearbeiten der Ontologie, das schließt das Laden, Hinzufügen und Löschen sowie das nötige Schlussfolgern mit ein. Die zweite Messung nimmt die Zeit zur Beantwortung der Abfrage.

Der u2r3 Schlussfolgerer wird immer in seiner Standardkonfiguration betrieben. Falls nicht ist dies angegeben.

Das Hinzufügen bei OWLim wurde so umgesetzt, dass man nochmal eine Ontologie mit dem neuen Teil geladen hat. Das Löschen wurde nicht durchgeführt.

Zunächst wird mit der walter-Ontologie [\ref{table-time-walter}] begonnen. Sie deckt einen Großteil der in OWL2 RL vorhandenen Sprachkonstrukte ab.

\begin{table}[htbp]
\caption{Zeiten für die walter-Ontologie}
\label{table-time-walter}
\begin{center}
\begin{threeparttable}
\begin{tabular}{l|r|r|r|r|r|r|}
\cline{2-7}
 & \multicolumn{2}{|c|}{Laden} & \multicolumn{2}{|c|}{Hinzufügen} & \multicolumn{2}{|c|}{Löschen} \\
\cline{2-7}
 & \multicolumn{1}{|c|}{Ableitung} & \multicolumn{1}{|c|}{Abfrage} & \multicolumn{1}{|c|}{Ableitung} & \multicolumn{1}{|c|}{Abfrage} & \multicolumn{1}{|c|}{Ableitung} & \multicolumn{1}{|c|}{Abfrage} \\
\hline
\multicolumn{1}{|l|}{u2r3\tnote{a}} & 17s & 27ms & 4s & 2ms & 100ms & 1ms \\ \hline
\multicolumn{1}{|l|}{u2r3\tnote{b}} & 81s & 28ms & 5s & 2ms & 1500ms & 3ms \\ \hline
\multicolumn{1}{|l|}{u2r3\tnote{c}} & 15s & 54ms & 5s & 2ms & 75ms & 1ms \\ \hline
\multicolumn{1}{|l|}{u2r3\tnote{d}} & 17s & 36ms & 4s & 3ms & 1215ms & 1ms \\ \hline
\multicolumn{1}{|l|}{u2r3\tnote{e}} & 16s & 35ms & 3s & 1ms & \multicolumn{1}{c|}{x} & \multicolumn{1}{c|}{x} \\ \hline
\multicolumn{1}{|l|}{HermiT} & 3s & 7ms & 0s & 1ms & 1ms & 2ms \\ \hline
\multicolumn{1}{|l|}{Pellet\tnote{f}} & 3s & 3ms & 0s & 1ms & 1ms & 1ms \\ \hline
\multicolumn{1}{|l|}{Racer\tnote{g}} & \multicolumn{1}{c|}{x} & \multicolumn{1}{c|}{x} & \multicolumn{1}{c|}{x} & \multicolumn{1}{c|}{x} & \multicolumn{1}{c|}{x} & \multicolumn{1}{c|}{x} \\ \hline
\multicolumn{1}{|l|}{Fact++\tnote{g}} & \multicolumn{1}{c|}{x} & \multicolumn{1}{c|}{x} & \multicolumn{1}{c|}{x} & \multicolumn{1}{c|}{x} & \multicolumn{1}{c|}{x} & \multicolumn{1}{c|}{x} \\ \hline
\multicolumn{1}{|l|}{OWLim} & 2s & 230ms & 2s & 245ms & \multicolumn{1}{c|}{x} & \multicolumn{1}{c|}{x} \\ \hline
\end{tabular}
\begin{tablenotes}
	\item[a] Standardkonfiguration
	\item[b] (CASCADING, STANDALONE, IMMEDIATE, DEFAULT)
	\item[c] (CASCADING, STANDALONE, COLLECTIVE, NONE)
	\item[d] (CASCADING, IN-MEMORY, COLLECTIVE, NONE)
	\item[e] (CLEAN, STANDALONE, COLLECTIVE, NONE)
	\item[f] Das Ergebnis ist aber nach dem Hinzufügen falsch.
	\item[g] Die Ontologie konnte nicht geparst werden. Das liegt wahrscheinlich daran, dass die Ontologie Elemente aus OWL2 verwendet die hier noch nicht unterstützt sind.
\end{tablenotes}
\end{threeparttable}
\end{center}
\end{table}

Das Ergebnis entspricht den Erwartungen. Die Vorbereitungszeit bei u2r3 ist zwar hoch. Die Abfragezeiten sind aber auf einem ähnlichen Niveau.
Man sieht, dass der immediate-Modus nicht geeignet ist für so eine Ontologie. Die Abfragezeiten von OWLim sind etwas hoch, das liegt vermutlich daran, das sein Backend erst bei größeren Ontologien effizient arbeitet.

Die DomusAG-Ontolgie [\ref{table-time-domusag}] benutzt Konstrukte, die in OWL2 RL nicht erlaubt sind. Mit einer Konfigurationsoption kann man aber extra Regeln in u2r3 aktivieren, damit er damit umgehen kann.
\begin{table}[htbp]
\caption{Zeiten für die DomusAG-Ontologie}
\label{table-time-domusag}
\begin{center}
\begin{threeparttable}
\begin{tabular}{l|r|r|r|r|r|r|}
\cline{2-7}
 & \multicolumn{2}{|c|}{Laden} & \multicolumn{2}{|c|}{Hinzufügen} & \multicolumn{2}{|c|}{Löschen} \\
\cline{2-7}
 & \multicolumn{1}{|c|}{Ableitung} & \multicolumn{1}{|c|}{Abfrage} & \multicolumn{1}{|c|}{Ableitung} & \multicolumn{1}{|c|}{Abfrage} & \multicolumn{1}{|c|}{Ableitung} & \multicolumn{1}{|c|}{Abfrage} \\
\hline
\multicolumn{1}{|l|}{u2r3} & 57s & 68ms & 1571ms & 28ms & 1ms & 78ms \\ \hline
\multicolumn{1}{|l|}{HermiT\tnote{a}} & \multicolumn{1}{c|}{x} & \multicolumn{1}{c|}{x} & \multicolumn{1}{c|}{x} & \multicolumn{1}{c|}{x} & \multicolumn{1}{c|}{x} & \multicolumn{1}{c|}{x} \\ \hline
\multicolumn{1}{|l|}{Pellet} & 37s & 13ms & 0ms & 2ms & 1ms & 2ms \\ \hline
\multicolumn{1}{|l|}{Racer} & 11s & 35ms & 2784ms & 11ms & 2587ms & 10ms \\ \hline
\multicolumn{1}{|l|}{Fact++\tnote{b}} & 5s & \multicolumn{1}{c|}{x} & \multicolumn{1}{c|}{x} & \multicolumn{1}{c|}{x} & \multicolumn{1}{c|}{x} & \multicolumn{1}{c|}{x} \\ \hline
\multicolumn{1}{|l|}{OWLim} & 4s & 400ms & 2516ms & 411ms & \multicolumn{1}{c|}{x} & \multicolumn{1}{c|}{x} \\ \hline
\end{tabular}
\begin{tablenotes}
	\item[a] Hermit bricht mit der Meldung \emph{unsupported datatype} ab.
	\item[b] Fact++ bricht mit der Meldung \emph{unknown class} ab, da die Ontologie mit Elementen arbeitet die erst in OWL2 vorhanden sind war dies zu erwarten
\end{tablenotes}
\end{threeparttable}
\end{center}
\end{table}
Das Ergebnis entspricht den Erwartungen.

Die VAST2009-tiny Ontologie [\ref{table-time-vast-tiny}] ist eime verkleinerte Version der VAST2009-Ontologie. Sie enthält eine große Anzahl an Beziehungen (data- und objectPropertyAssertion) und ist daher das angestrebte Einsatzgebiet von OWL2 RL und damit u2r3.

\begin{table}[htbp]
\caption{Zeiten für VAST2009-tiny Ontologie}
\label{table-time-vast-tiny}
\begin{center}
\begin{threeparttable}
\begin{tabular}{l|r|r|r|r|r|r|}
\cline{2-7}
 & \multicolumn{2}{|c|}{Laden} & \multicolumn{2}{|c|}{Hinzufügen} & \multicolumn{2}{|c|}{Löschen} \\
\cline{2-7}
 & \multicolumn{1}{|c|}{Ableitung} & \multicolumn{1}{|c|}{Abfrage} & \multicolumn{1}{|c|}{Ableitung} & \multicolumn{1}{|c|}{Abfrage} & \multicolumn{1}{|c|}{Ableitung} & \multicolumn{1}{|c|}{Abfrage} \\
\hline
\multicolumn{1}{|l|}{u2r3} & 11s & 26ms & 63ms & 4ms & 1ms & 2ms \\ \hline
\multicolumn{1}{|l|}{HermiT\tnote{a}} & 3s & 8ms & 1ms & 0ms & 1ms & 0ms \\ \hline
\multicolumn{1}{|l|}{Pellet\tnote{b}} & 3s & 2ms & 1ms & 1ms & 0ms & 1ms \\ \hline
\multicolumn{1}{|l|}{Racer\tnote{b}} & 2s & 2ms & 58ms & 2ms & 58ms & 1ms \\ \hline
\multicolumn{1}{|l|}{Fact++\tnote{c}} & 2s & 1ms & \multicolumn{1}{c|}{x} & \multicolumn{1}{c|}{x} & \multicolumn{1}{c|}{x} & \multicolumn{1}{c|}{x} \\ \hline
\multicolumn{1}{|l|}{OWLim} & 2s & 228ms & 1918ms & 243 & \multicolumn{1}{c|}{x} & \multicolumn{1}{c|}{x} \\ \hline
\end{tabular}
\begin{tablenotes}
	\item[a] Ergebnis nach dem Hinzufügen falsch
	\item[b] Der Reasoner konnte aber kein korrektes Ergebnis liefern.
	\item[c] Die Anwendung wurde vorzeitig beendet , da die JVM ein SIGSEV meldete.
\end{tablenotes}
\end{threeparttable}
\end{center}
\end{table}

Man sieht das die Zeiten hier nicht mehr weit auseinander liegen.

Die VAST2009-Ontologie [\ref{table-time-vast2009}] enthält jetzt dievolle Anzahl an Individuen und Beziehungen.

\begin{table}[htbp]
\caption{Zeiten für die VAST2009-Ontologie}
\label{table-time-vast2009}
\begin{center}
\begin{threeparttable}
\begin{tabular}{l|r|r|r|r|r|r|}
\cline{2-7}
 & \multicolumn{2}{|c|}{Laden} & \multicolumn{2}{|c|}{Hinzufügen} & \multicolumn{2}{|c|}{Löschen} \\
\cline{2-7}
 & \multicolumn{1}{|c|}{Ableitung} & \multicolumn{1}{|c|}{Abfrage} & \multicolumn{1}{|c|}{Ableitung} & \multicolumn{1}{|c|}{Abfrage} & \multicolumn{1}{|c|}{Ableitung} & \multicolumn{1}{|c|}{Abfrage} \\
\hline
\multicolumn{1}{|l|}{u2r3} & 4166s & 603ms & 1305s & 250ms & 179s & 201ms \\ \hline
\multicolumn{1}{|l|}{HermiT\tnote{a}} & \multicolumn{1}{c|}{x} & \multicolumn{1}{c|}{x} & \multicolumn{1}{c|}{x} & \multicolumn{1}{c|}{x} & \multicolumn{1}{c|}{x} & \multicolumn{1}{c|}{x} \\ \hline
\multicolumn{1}{|l|}{Pellet\tnote{b}} & \multicolumn{1}{c|}{x} & \multicolumn{1}{c|}{x} & \multicolumn{1}{c|}{x} & \multicolumn{1}{c|}{x} & \multicolumn{1}{c|}{x} & \multicolumn{1}{c|}{x} \\ \hline
\multicolumn{1}{|l|}{Racer\tnote{c}} & 2982s & 17ms & \multicolumn{1}{c|}{x} & \multicolumn{1}{c|}{x} & \multicolumn{1}{c|}{x} & \multicolumn{1}{c|}{x} \\ \hline
\multicolumn{1}{|l|}{Fact++\tnote{d}} & 12s & \multicolumn{1}{c|}{x} & \multicolumn{1}{c|}{x} & \multicolumn{1}{c|}{x} & \multicolumn{1}{c|}{x} & \multicolumn{1}{c|}{x} \\ \hline
\multicolumn{1}{|l|}{OWLim} & 1399s & 1346ms & 15s & 1575ms & \multicolumn{1}{c|}{x} & \multicolumn{1}{c|}{x} \\ \hline
\end{tabular}
\begin{tablenotes}
	\item[a] 700MB Heap waren zu wenig
	\item[b] 700MB Heap waren zu wenig
	\item[c] JVM Fehlermeldung
	\item[d] 700MB Heap waren zu wenig
\end{tablenotes}
\end{threeparttable}
\end{center}
\end{table}

Die Tableaux-basierten Reasoner arbeiten alle hauptsächlich im Arbeitsspeicher. Das zeigt sich bei einer solchen Ontologie ziemlich deutlich, da sie gegen Speicherbeschränkungen stoßen.

Die financial-Ontologie \ref{table-time-financial} ist ähnlich der VAST2009-Ontologie. Sie ist noch ein bißchen größer enthält allerdings nur objectPropertyAssertions und keine Literale.

\begin{table}[htbp]
\caption{Zeiten für die financial-Ontologie}
\label{table-time-financial}
\begin{center}
\begin{threeparttable}
\begin{tabular}{l|r|r|r|r|r|r|}
\cline{2-7}
 & \multicolumn{2}{|c|}{Laden} & \multicolumn{2}{|c|}{Hinzufügen} & \multicolumn{2}{|c|}{Löschen} \\
\cline{2-7}
 & \multicolumn{1}{|c|}{Ableitung} & \multicolumn{1}{|c|}{Abfrage} & \multicolumn{1}{|c|}{Ableitung} & \multicolumn{1}{|c|}{Abfrage} & \multicolumn{1}{|c|}{Ableitung} & \multicolumn{1}{|c|}{Abfrage} \\
\hline
\multicolumn{1}{|l|}{u2r3} & 5055s & 263ms & 2130s & 157ms & 84s & 298ms \\ \hline
\multicolumn{1}{|l|}{HermiT\tnote{a}} & \multicolumn{1}{c|}{x} & \multicolumn{1}{c|}{x} & \multicolumn{1}{c|}{x} & \multicolumn{1}{c|}{x} & \multicolumn{1}{c|}{x} & \multicolumn{1}{c|}{x} \\ \hline
\multicolumn{1}{|l|}{Pellet\tnote{b}} & 128s & 21ms & 0s & 23ms & 0s & 1104ms \\ \hline
\multicolumn{1}{|l|}{Racer\tnote{c}} & 112s & 2s & \multicolumn{1}{c|}{x} & \multicolumn{1}{c|}{x} & \multicolumn{1}{c|}{x} & \multicolumn{1}{c|}{x} \\ \hline
\multicolumn{1}{|l|}{Fact++} & 71s & 8ms & 55s & 8ms & 63s & 7ms \\ \hline
\multicolumn{1}{|l|}{OWLim} & 16 & 1121ms & 3s & 1124ms & \multicolumn{1}{c|}{x} & \multicolumn{1}{c|}{x} \\ \hline
\end{tabular}
\begin{tablenotes}
	\item[a] HermiT wurde nach zwei Stunden ohne Antwort abgebrochen
	\item[b] Das Ergebnis nach dem Hinzufügen war falsch
	\item[c] Nach dem Hinzufügen abgestürzt, da kein Heap-Space mehr vorhanden war (700MB).
\end{tablenotes}
\end{threeparttable}
\end{center}
\end{table}

HermiT hat zwar diesmal keinen so hohen Speicherverbrauch, muss aber nach zwei Stunden ohne Antwort abgebrochen werden. OWLim scheint mit dieser Ontologie wesentlich besser umgehen zu können.

Die instrument-Ontologie \ref{table-time-instrument} ist eine kleinere bis mittlere Ontologie, die auschließlich eine Klassenhierarchie enthält.

\begin{table}[htbp]
\caption{Zeiten für die instrument-Ontologie}
\label{table-time-instrument}
\begin{center}
\begin{threeparttable}
\begin{tabular}{l|r|r|r|r|r|r|}
\cline{2-7}
 & \multicolumn{2}{|c|}{Laden} & \multicolumn{2}{|c|}{Hinzufügen} & \multicolumn{2}{|c|}{Löschen} \\
\cline{2-7}
 & \multicolumn{1}{|c|}{Ableitung} & \multicolumn{1}{|c|}{Abfrage} & \multicolumn{1}{|c|}{Ableitung} & \multicolumn{1}{|c|}{Abfrage} & \multicolumn{1}{|c|}{Ableitung} & \multicolumn{1}{|c|}{Abfrage} \\
\hline
\multicolumn{1}{|l|}{u2r3} & 22s & 11ms & 294ms & 2ms & 1ms & 2ms \\ \hline
\multicolumn{1}{|l|}{HermiT\tnote{a}} & 5s & 9ms & 1ms & 1ms & 1ms & 1ms \\ \hline
\multicolumn{1}{|l|}{Pellet\tnote{b}} & 3s & 14ms & 1ms & 7ms & 1ms & 1ms \\ \hline
\multicolumn{1}{|l|}{Racer\tnote{c}} & 3s & 1ms & 215ms & 1ms & 397ms & \multicolumn{1}{c|}{x} \\ \hline
\multicolumn{1}{|l|}{Fact++\tnote{c}} & 2s & 1ms & 1ms4 & 1ms & 13ms & \multicolumn{1}{c|}{x} \\ \hline
\multicolumn{1}{|l|}{OWLim} & 3s & 276ms & 1650s & 294ms & \multicolumn{1}{c|}{x} & \multicolumn{1}{c|}{x} \\ \hline
\end{tabular}
\begin{tablenotes}
	\item[a] nach dem Hinzufügen falsch
	\item[b] aber nach Hinzufügen falsch
	\item[c] Stürtzt ab, da die gesuchte Klasse nicht gefunden wird, das stimmt zwar ist aber ein komisches Verhalten.
\end{tablenotes}
\end{threeparttable}
\end{center}
\end{table}

Das Ergebnis entspricht den Erwartungen. Man sieht das die Abfragezeit von u2r3 mit den anderen Schlussfolgerern auf gleicher Höhe ist.

Die disease-Ontologie \ref{table-time-disease} ist ähnlich der instrument-Ontologie nur wesentlich größer.
\begin{table}[htbp]
\caption{Zeiten für die disease-Ontologie}
\label{table-time-disease}
\begin{center}
\begin{threeparttable}
\begin{tabular}{l|r|r|r|r|r|r|}
\cline{2-7}
 & \multicolumn{2}{|c|}{Laden} & \multicolumn{2}{|c|}{Hinzufügen} & \multicolumn{2}{|c|}{Löschen} \\
\cline{2-7}
 & \multicolumn{1}{|c|}{Ableitung} & \multicolumn{1}{|c|}{Abfrage} & \multicolumn{1}{|c|}{Ableitung} & \multicolumn{1}{|c|}{Abfrage} & \multicolumn{1}{|c|}{Ableitung} & \multicolumn{1}{|c|}{Abfrage} \\
\hline
\multicolumn{1}{|l|}{u2r3} & 613s & 6ms & 159s & 3ms & 1ms & 1ms \\ \hline
\multicolumn{1}{|l|}{HermiT\tnote{a}} & 292s & 8ms & 0s & 15ms & 0ms & 1ms \\ \hline
\multicolumn{1}{|l|}{Pellet\tnote{a}} & 27s & 175ms & 0s & 1ms & 1ms & 1ms \\ \hline
\multicolumn{1}{|l|}{Racer\tnote{b}} & \multicolumn{1}{c|}{x} & \multicolumn{1}{c|}{x} & \multicolumn{1}{c|}{x} & \multicolumn{1}{c|}{x} & \multicolumn{1}{c|}{x} & \multicolumn{1}{c|}{x} \\ \hline
\multicolumn{1}{|l|}{Fact++\tnote{c}} & 10s & 1ms & 1s & 1ms & 503ms & \multicolumn{1}{c|}{x} \\ \hline
\multicolumn{1}{|l|}{OWLim} & 22s & 783ms & 3s & 768ms & \multicolumn{1}{c|}{x} & \multicolumn{1}{c|}{x} \\ \hline
\end{tabular}
\begin{tablenotes}
	\item[a] Das Ergebnis nach dem Hinzufügen ist aber falsch.
	\item[b] Fact++ endet mit der Fehlermeldung \emph{TokenMgrError: Lexical error}.
	\item[c] Stürtzt ab, da die gesuchteKlasse nicht mehr vorhanden ist. Das ist zwar korrekt aber seltsam.
\end{tablenotes}
\end{threeparttable}
\end{center}
\end{table}

Der relative Abstand zwischen u2r3 und HermiT ist stark geschrumpft. OWLim scheint mit solchen Ontologien sehr gut umgehen zu können.

\subsection{Zusammenfassung}
Die Reasoner die über die alte OWLAPIv2 angesprochen worden sind scheinen nicht mit den Ontologien klar gekommen zu sein. Ob es an Fehlern in den Schlussfolgerern, an der OWLAPI oder an den Ontologien selbst lag konnte nicht ermittelt werden. Das schränkt den Vergleich sehr ein. HermiT der einzige verbleibende Reasoner wurde über die selbe API wie u2r3 angesprochen. Er konnte alle Aufgaben mit Ausnahme der großen Ontologien VAST2009 und financial lösen. Das war aber auch zu erwearten. Unerwartet war, dass das spätere Hinzufügen von Axiomen nicht korrekt abgeleitet werden konnte. OWLim hat sich als zuverlässig herausgestellt, da es jede Ontologie und Anfrage bearbeiten konnte.

Im Hinblick auf u2r3 bedeutet dieses Ergebnis, das er zwar langsam beim Laden ist, dafür aber auch mit großen Ontologien umgehen kann. Die Anfragezeitn sind konstand in einem entsprechend kleinen Zeitfenster gelieben. Außerdem werden Hinzufügungen und Löschungen entsprechend ihrere Größe schnell behandelt ohne die Ontologie neu laden zu müssen.

Um einen bessern Vergleich zu erhalten sollt man diesen Test noch einaml wiederholen, wenn die Schnittstellen der Schlussfolgerer alle auf der OWLAPIv3 basieren. Trotzdem sollten diese Tests einen Überblick über den Anwendungsbereich des u2r3 geschaffen haben.

