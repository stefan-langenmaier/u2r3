\section{Aufbau des Schlussfolgerers}



\tikzstyle{composition}=[->, >=diamond, thick]

\tikzstyle{class}=[rectangle, draw=black, rounded corners, fill=white, drop shadow, text centered, anchor=north, text width=3.5cm, rectangle split, rectangle split parts=1]

\begin{tikzpicture}[node distance=2cm]
    \node (U2R3Reasoner) [class]
        {
            \textbf{U2R3Reasoner}
%            \nodepart{second}name
        };
        
    \node (RuleManager) [class, below of= U2R3Reasoner]
        {
            \textbf{RuleManager}
%            \nodepart{second}name
        };
        
     \node (RelationManager) [class, left =0.1cm of RuleManager]
        {
            \textbf{RelationManager}
%            \nodepart{second}name
        };
        
    \node (ReasonProcessor) [class, rectangle split parts=2, right =0.1cm of RuleManager]
        {
            \textbf{ReasonProcessor}
            \nodepart{second}+ classify()
        };
        
    \draw[composition] (RuleManager.north) -- (U2R3Reasoner.south);
    \draw[composition] (RelationManager.north) -- (U2R3Reasoner.south);
    \draw[composition] (ReasonProcessor.north) -- (U2R3Reasoner.south);
        
        
\end{tikzpicture}
        