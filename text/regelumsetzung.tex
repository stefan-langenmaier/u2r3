\section{Regelumsetzung allgemein}
\label{abschnitt-regelumsetzung}
Regeln sind hier sogenannte Inferenzprozeduren, das sind automatisierte Verfahren zur Berechnung logischer Folgerungen. Inferenzprozeduren sind bereits aus der Aussagenlogik bekannt.

\begin{table}[htb]
\begin{center}
	\begin{tabular}{l}
	A$\rightarrow$B \\
	A \\
	\hline
	B
	\end{tabular}
\end{center}
	\caption{Modus ponens}
	\label{table-modus-ponens}
\end{table}

\begin{table}[htb]
\begin{center}
	\begin{tabular}{l}
	A$\rightarrow$B \\
	$\neg$B \\
	\hline
	$\neg$A
	\end{tabular}
\end{center}
	\caption{Modus tollens}
	\label{table-modus-tollens}
\end{table}

Diese beiden Ableitungsregeln [\ref{table-modus-ponens}, \ref{table-modus-tollens}] erlauben es, die beiden Aussagen in der Prämisse durch die Schlussfolgerung zu ersetzen. Für einen Schlussfolgerer mit direct-materialization werden, die Prämissen nicht ersetzt sondern weiterhin erhalten. Es könnte ja möglich sein das weitere Ableitungsschritte damit möglich sind oder das spätere Anfragen auf diese Aussagen abzielen. Mit einer Ableitungsregel kann also die Aussagenbasis vergrößert werden.

Es sind dabei jedoch nicht alle Ableitungsregeln gültig.
\begin{table}[htb]
\begin{center}
	\begin{tabular}{l}
	A$\rightarrow$B \\
	B \\
	\hline
	A
	\end{tabular}
\end{center}
	\caption{ungültige Ableitung}
	\label{table-invalid-inferred}
\end{table}
Abbildung \ref{table-invalid-inferred} ist z.B. kein gültiger Schluss.

Eine Menge von vernünftigen Ableitungsregeln zu finden ist keine leichte Aufgabe. Dabei eine möglichst große Ausdrucksmächtigkeit zu erhalten, ohne die Komplexität der Regeln aus den Augen zu verlieren wurde versucht mit dem OWL2 RL Fragement zu erreichen.

\subsection{Regelbeispiele}
In diesem Teil der Regelumsetzung wird an konkreten Beispielen gezeigt, wie eine Umsetzung der OWL2 RL Regeln in SQL möglich ist. Die SQL Abfragen benutzen dabei die vorher im MEMA-Prinzip erstellen Relationen

Die Abfragen werden zunächst allgemein an einfachen Regeln prinzipiell erklärt, danach wird auf die Sonderfälle eingegangen.

Insgesamt kann dabei in vier Kategorie unterschieden werden:
\begin{itemize}
  \item Gewöhnliche Regeln: Dabei werden verschiedene Fakten miteinander verknüpft, so das dabei neue Fakten entstehen können. Diese Regeln sind alle sehr schematisch umsetzbar.
  \item Listenregeln: Hier werden Listen von Fakten bearbeitet. Dabei ist nicht klar wie viele Elemente eine Liste enthält. Dies kann zu einem Problem werden, wenn man die Entstehungsgeschichte von Fakten, z.B. für das Löschen mitabspeichern will. Hier muss man gesondert darauf achten das diese Information der Entstehung nicht verloren geht.
  \item Inkonsistenzregeln: Sie sind ähnlich der neuen Regeln, erzeugen allerdings keine neuen Fakten. Falls sie neue Fakten erzeugen könnten bedeutet dies eine Inkonsistenz in der Ontologie.
  \item Regeln für Datentypen: Für typisierte Literal werden einige Überprüfung bzgl. der Gleichheit untereindander und der Konformität zu den in OWL2 RL eingebauten Datentypen durchgeführt.
  \item Einmalige Regeln: Einige Regeln haben keine Vorbedigung. Diese werden einmal zu Beginn der Regelanwendung ausgeführt.
\end{itemize}

\subsubsection{Einmalige Regeln}

cls-thing
if | then
true | owl:Thing rdf:type owl:Class

Diese Regel wird einmal zu Beginn in die Liste der Regelanwendungen gesteckt. Sie sorgt dafür das dieses Faktum eingefügt wird.

In SQL lautet dies:

INSERT INTO classAssertionEnt (entity, class)
	VALUES ('owl:Thing', 'owl:Class')

\subsubsection{Inkonsistenzregeln}
Inkonsistenzregeln erzeugen keine Fakten. Sie versuchen aber gewisse Fakten zu finden, die im Widerspruch zu einander stehen.

Die Inkonsistenzregeln cls-nothing2 sieht dabei so aus:

If | then
T(?x, rdf:type, owl:Nothing) | false

Dies wird in folgende SQL-Abfrage umgewandelt:

SELECT 1
FROM classAssertionEnt
WHERE class = 'owl:Nothing'

Die Relation classAssertionEnt ist die Relation, die die type-Beziehung speichert. Ist darin eine Zeile zu finden, die als type die KLasse owl:Nothing hat wird eine Zeile zurückgegeben. Falls also diese Abfrage eine oder mehrere Zeilen erzeugt liegt eine Inkonsistenz vor.

\subsubsection{Gewöhnliche Regeln}
Gewöhnliche Regeln erzeugen neue Fakten in der Datenbank. Eine einfache Regel ist hier eq-sym:

If | then
T(?x, owl:sameAS, ?y) | T(?y, owl:sameAS, ?x)

Dies wird wie folgt in SQL überführt:

INSERT INTO sameAs (left, right)
SELECT right, left
FROM sameAs

Wie die Abfrage zu Stande kommt sollte klar sein. Allerdings wurden in dieser Umwandlung schon einge Dinge vereinfacht, die in der Implementierung so nicht gemacht wurden.

\begin{enumerate}
  \item Was passiert wenn eine Zeile eingefügt werden soll, die schon enthalten ist?
  \item Wie würde man hier die Delta-Iteration einsetzen können, um nicht immer auf alle Fakten schließen zu müssen?
  \item Wie kann man die Entstehungsgeschiechte von neuen Fakten mitschreiben, um später effizientes Löschen zu ermöglichen.
\end{enumerate}

Diese Punkte werden jeweils in ihren speziellen Abgeschnitten. Um die Beispiel nicht untnötig komplizierter zu machen wird hier nicht näher darauf eingegangen.

Ein komplexeres Beispiel ist die Verknüpfung von Fakten um auf neue Fakten schließen zu können, z.B. in der Regel eq-rep-s:

If | then
T(?s1, owl:sameAs, ?s2), T(?s1, ?p, ?o) | T(?s2, ?p, ?o)

Die Verknüpfung wird durch einen JOIN auf die entsprechende Spalte realisiert.

INSERT INTO objectPropertyAssertion(subject, object, property)
SELECT sa.right, opa.property, opa.object
FROM objectPropertyAssertion AS opa
INNER JOIN sameAs AS sa
	ON sa.left = opa.subject

\section{Delta-Iteration}
\label{abschnitt-delta-iteration}

Unabhängig von der Art der delta-Iteration muss immer der Ausführungskontext bekannt sein, d.h. welche delta-Iteration ist aktiv, auf welches delta wird gearbeitet. Ein delta muss dabei die Information enthalten auf welcher Relation es arbeitet und auf welchen Teil. Zum Ausführungskontext gehört je nach Modus auch noch die Information, ob eine Regel schon auf das Ziel delta angewendet wurde.

Die delta-Iteration ist ein Verfahren, um bei Regelanwendungen, die Datenmenge so zu reduzieren, das nur solche enthalten sind, die noch nicht berücksichtigt wurden und somit die tatsächlich zu betrachtenden Menge zu reduzieren.

Während der ersten Entwicklung sind dabei zwei Varianten der delta-Iteration klar geworden. Diese tragen im folgenden die Namen \emph{immediate} delta-Iteration und \emph{collective} delta-Iteration. Timo Weithöhner hat in seiner Implementierung die collective delta-Iteration verwendet.

\subsection{Immediate delta-Iteration}

Hierbei erzeugt jede Regelanwendung sofort ihr eigenes Delta, mit dem weitergearbeitet werden kann. Wenn Zeilen erzeugt wurden, wird sofort das dazugehörige Delta erzeugt und dafür alle notwendigen Regelanwendungen ausgelöst.

Im folgenden ist ein Beispiel gegeben, wie der Ablauf bei der Anwendung einer Regel ist die neue Fakten erzeugt hat und wie diese dann verarbeitet wird. Angenommen die Regel \emph{eq-ref-s} erzeugt neue Fakten, diese landen in einem Delta der \emph{sameas}-Relation. Das löst eine ``Reason'' aus mit der Information welches Delta neue Fakten enthalten hat. Wenn der Regelprozessor diese erhält erzeugt er für diese Kombination sog. RuleActions, d.h. er bringt alle Regeln die auf diesem delta arbeiten in eine Warteschlange. Zu jeder Regel speichert er den Ausführungskontext, auf den sie angewendet werden soll mit ab. Die Regelanwendungen warten dann bis sie zur Ausführung gebracht werden.

\begin{verbatim}
Regel: eq-ref-s //neue Fakten
=> delta sameas
   => new Reason(sameas, delta)
      => new RuleAction(eq-sym, delta)
         new RuleAction(eq-trans, delta)
         new RuleAction(eq-rep-s, delta)
         new RuleAction(eq-rep-p, delta)
         new RuleAction(eq-rep-o, delta)
         new RuleAction(eq-diff1, delta)
         new RuleAction(eq-diff2, delta)
         new RuleAction(eq-diff3, delta)
\end{verbatim}

Der Regelprozessor ist dann auch verantwortlich die Ausführung der Regelanwendungen zu starten. Das passiert in einer Schleife, die die Elemente der Warteschlange abarbeitet bis keine weiteren Elemente vorhanden sind.

Dies ist im folgenden Code \ref{code-immediate-delta-iteration} schematisch dargestellt:

\begin{figure}[htp]
	\caption{Abarbeitung der RuleActions im immediate Modus.}
	\label{code-immediate-delta-iteration}
	\begin{lstlisting}[language=Java]
while(action = rulesToApply.popAction()) {
	action.rule.apply(action.delta);
	
	if (!(rulesToApply.contains(delta)) {
		delta.addToRelation();
		delta.drop();
		delta == null;
	}
}
	\end{lstlisting}
\end{figure}


Der Vorteil dieser Vorgehensweise ist das es besser parallelisiert werden kann, da soviele Actions wie möglich gleizeitig angewand werden können. Die Parallelisierung ist eines der Ziele die eventuell in der Zukunft umgesetzt werden sollen.

Der Nachteil ist das viele Deltas gleichzeitg entstehen und insgesamt mehr Regel angewendet werden müssen. Die Deltas enthalten unter Umständen nur wenig Inhalt, d.h. die Abfragen arbeiten auf kleineren Datenmenge aber ein Datenbanksystem lohnt sich eventuell erst wirklich bei großen Datenmengen. Es muss nicht zwangsläufig schlecht sein, aber es ist nicht unbedingt klar wie sich das auf ein DBMS auswirkt.

\subsection{Collective delta-Iteration}

Vom aktuellen Stand einer Relation werden erst einmal alle Änderungen in einer Hilfsrelation gesammelt. Sind alle abgearbeitet, wird das Delta erstellt und die nächste Phase beginnt.

\begin{verbatim}
Regel: sameas-eq-ref-s //neue Fakten
=> aux sameas
\end{verbatim}

\begin{figure}[htp]
	\caption{Abarbeitung der RuleActions im immediate Modus.}
	\label{code-immediate-delta-iteration}
	\begin{lstlisting}[language=Java]
do {
	while (action = list.popAction()) {
		action.rule.apply(action.delta);
	}
}
while (applyRelations()); //WAIT --- SYNC

boolean applyRelations() {
	for(r : relations)
		r.applyAux();
		return relations.wereDirty()
}

applyAux() {
	createDelta;
	addDelta();
	clearAux();
	new Reason(relation, delta);
}
	\end{lstlisting}
\end{figure}

Vorteile:
\begin{itemize}
  \item Es gibt immer nur eine Hilfsrelation und ein Delta, also eine fixe Anzahl an Tabellen.
  \item Es werden evtl. größere Menge an neuen Fakten zusammengefasst. Das muss nicht notwendiger weise gut sein.
\end{itemize}

Ein Nachteil ist, das die Parallelsierung nicht vollständig durchgeführt werden kann, da zu gewissen Zeitpunkten (Zeile 6), die Regelanwendung unterbrochen wird und die Ausführung wieder synchronisiert werden muss.

\subsection{Simulation}

Die collective delta-Iteration kann mit Hilfe der immediate delta-Iteration und ein paar Änderungen umgesetzt werden. Dazu dürfen die Relationen nicht sofort neue Hinweise an den Regelprozessor schicken, sondern es gibt eine eigene Phase nach dem Abarbeiten aller Regeln. Dabei wird erst der Inhalt der Deltas in die Haupttabelle übernommen und es werden die Reasons an den Regelprozessor geschickt.

Die Hilfstabelle in der die Daten für die Runde zwischengespeichert werden ist das Delta, das neu angelegt wirde. Die alte Hilfstabelle wird zum Delta der aktuellen Runde.

Die Implementierung des u2r3 unterstützt beide Varianten der delta-Iteration. Diese können in der Konfigurationsdatei mit der Option DeltaIteration verändert werden. Gültige Werte sind \emph{COLLECTIVE} oder \emph{IMMEDIATE}.

