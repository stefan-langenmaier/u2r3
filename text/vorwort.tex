\section*{Vorwort}
Diese Arbeit beschäftigt sich mit der Implementierung und Optimierung eines relationalen Schlussfolgerers. Dieser Schlussfolgerer arbeitet dabei auf dem OWL2 RL Profil, das Ende Oktober 2009 vom W3C \cite{OWL2Profiles} veröffentlicht wurde. Dieser \emph{technical report} stellt ein klare Spezifikation der Semantik und der möglichen Schlussfolgerungsregeln in einer Ontologie dar. Da die Veröffentlichung der Spezifikation des W3C öffentlich ist und kompatibel zur Vorgängerversion OWL1 \cite{OWL1} ist steht es einer breiten Basis von Anwendungen und Anwendern zur Verfügung. Allerdings sind konkrete Umsetzungen von Software für OWL2 auf Grund seiner aktuellen Natur noch nicht vorhanden bzw im Moment in der Entwicklung. In dieser Arbeit wird daher auf teilweise noch experimenteller Software, wie z.B. der OWLAPI v3 \cite{OWLAPI}, aufgesetzt. Dies ist nötig, um für die Zukunft eine aktuelle Schnittstelle für Ontologien zur Verfügung zu stellen.

Vom Ansatz wie ein solcher Reasoner implementiert werden kann wird dabei in einigen Teilen auf die Arbeit von Timo Weithöhner zurückgegriffen. Sie ist damit auch eine Fortsetzung der Entwicklung des U2R2 (Uni Ulm Relational Reasoner), sowie eine Weiterführung, Überprüfung und Aktualisierung der Diplomarbeit "Speicherung und Abfrage großer Ontologien in deduktiven Datenbanken" aus dem Oktober 2003. Es wird daher angenommen, das der Leser zumindest mit den Erkenntnissen und Ergebnissen dieser Arbeit vertraut ist.

Es wird dabei im folgenden ein relationaler Reasoner mit forward-chaining und direct materialisation entwickelt. Relational steht dabei dafür das die Ergebnisse und Fakten der Ontologien in Relationen in einer Datenbank abgespeichert werden. Forward-chaining bedeutet das neue Fakten wieder neue Fakten erzeugen und soweit geschlussfolgert wird bis keine neuen Fakten mehr erzeugt werden könnne. Direct-materialisation steht dafür, das zu Beginn alle Fakten erzeugt und abgespeichert werden, zur Anfrage Zeit liegen dann alle möglichen Antworten schon vor und müssen nur noch ausgegeben werden.

Durch diese Eigenschaften ist es einerseits möglich große Ontologien zu laden und sie nicht vollständig im Arbeitsspeicher halten zu müssen, sondern können in Sekundärspeicher durch ein Datenbanksystem verwaltet sein. Andererseits wird durch den \emph{direct - materialisation} Ansatz der Fokus auf die Beantwortung von Fragen an die Ontologie gelenkt. Durch eine Vorberechnung ist es sehr schnell möglich darauf zu antworten.

Damit sollte die grundlegende Arbeitsweise schon aufgezeigt sein und der Leser sollte wissen was ihn erwartet.