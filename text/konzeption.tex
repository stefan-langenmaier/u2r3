    * Was ist der Unterschied zu OWL1 DL/Lite?
    * Was sind Fähigkeiten, die der neue U2R2 haben soll?
          o Schlussfolgern, Konsistenzprüfung, Änderung, Löschung, OWLAPI Anbindung, Konformitätsüberprüfung auf OWL2 RL 
    * Wie konform und semantisch korrekt war U2R2?
    * Was ist der Unterschied in der Menge und Art der Regelimplementierung
          o Welche Techniken wurden verwendet
          o Wieviele wurden implementiert
                + Aussagekraft der Sprachen 
          o Wie komplex sind die implementierten Regeln? Wären alle Regeln aus u2r3 auch in u2r3 implementierbar gewesen, oder wie aufwändig wäre es gewesen diese zu implementieren?
                + Lohnt sich deshalb schon der Umstieg auf ein RDBMS? (einfachere Aufbau, standardisierte Funktionen, größere Funktionsumfang) 
    * Wie wird die Trennung von Klassen und Individuen im MEMA-Modell erhalten?
    
    
        * Welche Fragestellungen gibt es?
    * Wie hängen diese zusammen?
    * Wie können spätere Manipluationen durchgeführt werden?
    * Was ist der Aufwand nach einer Änderung (von Fakten oder Konzepten)?
    * Was ist die Ausdrucksmächtigkeit von OWL2 RL?
    * Was ist die Zeit und Speicher Komplexität von OWL2 RL?
    * Was ist der Unterschied zur Ausdrucksmächtigkeit von U2R2 (1)?
    * Welche Erkenntisse gibt es bereits?
    * Was sind die Fähigkeiten/Ausdrucksmächtigkeiten der verschiedenen Fragmente?
    * Was sind ihre Komplexitäten?
    * Was versteht man unter Truth-Maintenance? 


    * Wie muss dadurch über die entstandenen Daten geschloßen werden?
    * Was sind deduktive Datenbanken?
    * Können die SQL Statements generiert werden?
    * Wie bildet man Regeln allgemein auf SQL ab? RAP?
    * Kann man quadrat. Rekursion im auflösen?
    * Wie sieht das DB-Schema aus?
    * Kann man manche Ableitungsschritte zusammenfassen?
    * Was bringen rekursive Abfragen wirklich? (Es werden dadurch ja auch andere Regeln angestoßen und dann müssen die rekursiven wieder ran)
    * Wie wandelt man die Ontologie in INSERTs um? 
    
     