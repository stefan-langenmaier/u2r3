\tikzstyle{relation}=[rectangle, draw=black, rounded corners, fill=white, drop shadow, text justified, anchor=north, text=black, text width=4cm]

\begin{tikzpicture}[node distance=6cm]
	\node (history) [relation, rectangle split, rectangle split parts=2]{
			\textbf{history}
		\nodepart{second}
			\underline{id}: INT\newline
			\underline{table}: TABLE\newline
			\underline{sourceId}: INT\newline
			\underline{sourceTable}: TABLE
	};
	\node (list) [relation, rectangle split, rectangle split parts=2, left of= history]{
			\textbf{list}
		\nodepart{second}
			id: INT\newline
			\textbf{\underline{name}}: CLASS\newline
			\textbf{\underline{element}}: CLASS/IND
	};

\end{tikzpicture}

\begin{figure}
\caption{Liste der Relationen, für die auch Fakten erzeugt werden}
	\label{relations-for-which-data-is-created}
\tikzstyle{relation}=[rectangle, draw=black, rounded corners, fill=white, drop shadow, text justified, anchor=north, text=black, text width=5cm]
\begin{tikzpicture}[node distance=2.5cm]
	\node (subClass) [relation, rectangle split, rectangle split parts=2]{
			\textbf{subClass}
		\nodepart{second}
			\underline{id}: INT\newline
			\textbf{\underline{sub}}: CLASS\newline
			\textbf{\underline{super}}: CLASS
	};
	
	\node (equivalentClass) [relation, rectangle split, rectangle split parts=2, right =2cm of subClass]{
			\textbf{equivalentClass}
		\nodepart{second}
			\underline{id}: INT\newline
			\textbf{\underline{left}}: CLASS\newline
			\textbf{\underline{right}}: CLASS
	};
	
	\node (subProperty) [relation, rectangle split, rectangle split parts=2, below of= subClass]{
			\textbf{subProperty}
		\nodepart{second}
			\underline{id}: INT\newline
			\textbf{\underline{sub}}: PROPERTY\newline
			\textbf{\underline{super}}: PROPERTY
	};
	
	\node (equivalentProperty) [relation, rectangle split, rectangle split parts=2, right =2cm of subProperty]{
			\textbf{equivalentProperty}
		\nodepart{second}
			\underline{id}: INT\newline
			\textbf{\underline{left}}: CLASS\newline
			\textbf{\underline{right}}: CLASS
	};
	
	
	\node (classAssertionEnt) [relation, rectangle split, rectangle split parts=2, below of= subProperty]{
			\textbf{classAssertionEnt}
		\nodepart{second}
			\underline{id}: INT\newline
			\textbf{\underline{entity}}: IND/CLASS\newline
			\textbf{\underline{class}}: CLASS
	};
	
	\node (classAssertionLit) [relation, rectangle split, rectangle split parts=2, right =2cm of classAssertionEnt]{
			\textbf{classAssertionLit}
		\nodepart{second}
			\underline{id}: INT\newline
			\textbf{\underline{literal}}: LITERAL\newline
			\textbf{\underline{class}}: CLASS\newline
			\textbf{\underline{language}}: LANGUAGE
	};
	
	\node (sameAsEnt) [relation, rectangle split, rectangle split parts=2, below =2cm of classAssertionEnt]{
			\textbf{sameAsEnt}
		\nodepart{second}
			\underline{id}: INT\newline
			\textbf{\underline{left}}: IND\newline
			\textbf{\underline{right}}: IND
	};
	
	\node (sameAsLit) [relation, rectangle split, rectangle split parts=2, right =2cm of sameAsEnt, text width=6cm]{
			\textbf{sameAsLit}
		\nodepart{second}
			\underline{id}: INT\newline
			\textbf{\underline{left}}: IND\newline
			\textbf{\underline{right}}: IND\newline
			\textbf{\underline{left\_type}}: CLASS\newline
			\textbf{\underline{right\_type}}: CLASS\newline
			\textbf{\underline{left\_language}}: LANGUAGE\newline
			\textbf{\underline{right\_language}}: LANGUAGE
	};
	
	\node (objectPropertyAssertion) [relation, rectangle split, rectangle split parts=2, below =2cm of sameAsEnt]{
			\textbf{objectPropertyAssertion}
		\nodepart{second}
			\underline{id}: INT\newline
			\textbf{\underline{subject}}: IND\newline
			\textbf{\underline{property}}: PROPERTY\newline
			\textbf{\underline{object}}: IND
	};
	
	\node (dataPropertyAssertion) [relation, rectangle split, rectangle split parts=2, right =2cm of objectPropertyAssertion, text width=6cm]{
			\textbf{dataPropertyAssertion}
		\nodepart{second}
			\underline{id}: INT\newline
			\textbf{\underline{subject}}: IND\newline
			\textbf{\underline{property}}: PROPERTY\newline
			\textbf{\underline{object}}: LITERAL\newline
			\textbf{\underline{type}}: CLASS\newline
			\textbf{\underline{language}}: LANGUAGE
	};
	
	 
	\node (propertyDomain) [relation, rectangle split, rectangle split parts=2, below= 1cm of objectPropertyAssertion]{
			\textbf{propertyDomain}
		\nodepart{second}
			\underline{id}: INT\newline
			\textbf{\underline{property}}: CLASS\newline
			\textbf{\underline{domain}}: CLASS
	};
	
	\node (propertyRange) [relation, rectangle split, rectangle split parts=2, right =2cm of propertyDomain]{
			\textbf{propertyRange}
		\nodepart{second}
			\underline{id}: INT\newline
			\textbf{\underline{property}}: CLASS\newline
			\textbf{\underline{range}}: CLASS
	};

\end{tikzpicture}
\end{figure} 

\begin{tikzpicture}[node distance=3cm]
	\node (members) [relation, rectangle split, rectangle split parts=2]{
			\textbf{members}
		\nodepart{second}
			id: INT\newline
			\textbf{\underline{class}}: CLASS\newline
			\textbf{\underline{list}}: NAME
	};
	
	\node (propertyChain) [relation, rectangle split, rectangle split parts=2, right =2cm of members]{
			\textbf{propertyChain}
		\nodepart{second}
			id: INT\newline
			\textbf{\underline{property}}: PROPERTY\newline
			\textbf{\underline{list}}: NAME
	};
	
	\node (hasKey) [relation, rectangle split, rectangle split parts=2, below of= members]{
			\textbf{hasKey}
		\nodepart{second}
			id: INT\newline
			\textbf{\underline{class}}: CLASS\newline
			\textbf{\underline{list}}: NAME
	};
	
	\node (intersectionOf) [relation, rectangle split, rectangle split parts=2, right =2cm of hasKey]{
			\textbf{intersectionOf}
		\nodepart{second}
			id: INT\newline
			\textbf{\underline{class}}: CLASS\newline
			\textbf{\underline{list}}: NAME
	};
	
	\node (unionOf) [relation, rectangle split, rectangle split parts=2, below of= hasKey]{
			\textbf{unionOf}
		\nodepart{second}
			id: INT\newline
			\textbf{\underline{class}}: CLASS\newline
			\textbf{\underline{list}}: NAME
	};
	
	
	\node (oneOf) [relation, rectangle split, rectangle split parts=2, right= 2cm of unionOf]{
			\textbf{oneOf}
		\nodepart{second}
			id: INT\newline
			\textbf{\underline{class}}: CLASS\newline
			\textbf{\underline{list}}: NAME
	};

\end{tikzpicture}


\tikzstyle{relation}=[rectangle, draw=black, rounded corners, fill=white, drop shadow, text justified, anchor=north, text=black, text width=6cm]
\begin{tikzpicture}[node distance=3cm]
	\node (allValuesFrom) [relation, rectangle split, rectangle split parts=2]{
			\textbf{allValuesFrom}
		\nodepart{second}
			id: INT\newline
			\textbf{\underline{part}}: CLASS\newline
			\textbf{\underline{property}}: CLASS\newline
			\textbf{\underline{total}}: NAME
	};
	
	\node (someValuesFrom) [relation, rectangle split, rectangle split parts=2, right =1cm of allValuesFrom]{
			\textbf{someValuesFrom}
		\nodepart{second}
			id: INT\newline
			\textbf{\underline{part}}: CLASS\newline
			\textbf{\underline{property}}: CLASS\newline
			\textbf{\underline{total}}: NAME
	};
	
	\node (hasValueEnt) [relation, rectangle split, rectangle split parts=2, below =1cm of allValuesFrom]{
			\textbf{hasValueEnt}
		\nodepart{second}
			id: INT\newline
			\textbf{\underline{class}}: CLASS\newline
			\textbf{\underline{property}}: PROPERTY\newline
			\textbf{\underline{value}}: IND
	};
	
	\node (hasValueLit) [relation, rectangle split, rectangle split parts=2, right =1cm of hasValueEnt]{
			\textbf{hasValueLit}
		\nodepart{second}
			id: INT\newline
			\textbf{\underline{class}}: CLASS\newline
			\textbf{\underline{property}}: PROPERTY\newline
			\textbf{\underline{value}}: LITERAL\newline
			\textbf{\underline{language}}: LANGUAGE\newline
			\textbf{\underline{type}}: CLASS
	};
	
	\node (disjointWith) [relation, rectangle split, rectangle split parts=2, below =1cm of hasValueEnt]{
			\textbf{disjointWith}
		\nodepart{second}
			id: INT\newline
			\textbf{\underline{left}}: CLASS\newline
			\textbf{\underline{right}}: CLASS
	};
	
	\node (propertyDisjointWith) [relation, rectangle split, rectangle split parts=2, right =1cm of disjointWith]{
			\textbf{propertyDisjointWith}
		\nodepart{second}
			id: INT\newline
			\textbf{\underline{left}}: PROPERTY\newline
			\textbf{\underline{right}}: PROPERTY
	};
	
	\node (maxCardinality) [relation, rectangle split, rectangle split parts=2, below of= disjointWith]{
			\textbf{maxCardinality}
		\nodepart{second}
			id: INT\newline
			\textbf{\underline{class}}: CLASS\newline
			\textbf{\underline{property}}: PROPERTY\newline
			\textbf{\underline{value}}: NUMBER
	};
	
	\node (maxQualifiedCardinality) [relation, rectangle split, rectangle split parts=2, right =1cm of maxCardinality]{
			\textbf{maxQualifiedCardinality}
		\nodepart{second}
			id: INT\newline
			\textbf{\underline{class}}: CLASS\newline
			\textbf{\underline{property}}: PROPERTY\newline
			\textbf{\underline{total}}: CLASS\newline
			\textbf{\underline{value}}: NUMBER
	};
	
	\node (complementOf) [relation, rectangle split, rectangle split parts=2, below of= maxCardinality]{
			\textbf{complementOf}
		\nodepart{second}
			id: INT\newline
			\textbf{\underline{left}}: CLASS\newline
			\textbf{\underline{right}}: CLASS
	};
	
	\node (inverseOf) [relation, rectangle split, rectangle split parts=2, right =1cm of complementOf]{
			\textbf{inverseOf}
		\nodepart{second}
			id: INT\newline
			\textbf{\underline{left}}: CLASS\newline
			\textbf{\underline{right}}: CLASS
	};
	
	\node (negativeObjectPropertyAssertion) [relation, rectangle split, rectangle split parts=2, below of= complementOf]{
			\textbf{negativeObjectPropertyAssertion}
		\nodepart{second}
			id: INT\newline
			\textbf{\underline{subject}}: IND\newline
			\textbf{\underline{property}}: PROPERTY\newline
			\textbf{\underline{object}}: IND
	};
	
	\node (negativeDataPropertyAssertion) [relation, rectangle split, rectangle split parts=2, right =1cm of negativeObjectPropertyAssertion]{
			\textbf{negativeDataPropertyAssertion}
		\nodepart{second}
			id: INT\newline
			\textbf{\underline{subject}}: IND\newline
			\textbf{\underline{property}}: PROPERTY\newline
			\textbf{\underline{object}}: IND
	};
	
\end{tikzpicture}


\begin{tikzpicture}[node distance=6cm]
	\node (differentFromEnt) [relation, rectangle split, rectangle split parts=2]{
			\textbf{differentFromEnt}
		\nodepart{second}
			id: INT\newline
			\textbf{\underline{left}}: IND\newline
			\textbf{\underline{right}}: IND
	};
	\node (differentFromLit) [relation, rectangle split, rectangle split parts=2, left of= differentFromEnt]{
			\textbf{differentFromLit}
		\nodepart{second}
			id: INT\newline
			\textbf{\underline{left}}: IND\newline
			\textbf{\underline{right}}: IND\newline
			left\_language: LANGUAGE\newline
			left\_type: CLASS\newline
			right\_language: LANGUAGE\newline
			right\_type: CLASS
	};
	
	\node (disjointWith) [relation, rectangle split, rectangle split parts=2, below of= differentFromEnt]{
			\textbf{disjointWith}
		\nodepart{second}
			id: INT\newline
			\textbf{\underline{left}}: CLASS\newline
			\textbf{\underline{right}}: CLASS
	};
	\node (propertyDisjointWith) [relation, rectangle split, rectangle split parts=2, left of= disjointWith]{
			\textbf{propertyDisjointWith}
		\nodepart{second}
			id: INT\newline
			\textbf{\underline{left}}: PROPERTY\newline
			\textbf{\underline{right}}: PROPERTY
	};

\end{tikzpicture}
