\section*{Zusammenfassung}
Diese Arbeit beschäftigt sich mit der Implementierung und Optimierung eines relationalen Schlussfolgerers. Dieser Schlussfolgerer arbeitet dabei auf dem OWL2 RL Profil, das Ende Oktober 2009 vom W3C \cite{OWL2Profiles} veröffentlicht wurde. Dieser \emph{technical report} stellt eine klare Spezifikation der Semantik und der möglichen Schlussfolgerungsregeln in einer Ontologie dar. Die Veröffentlichung der Spezifikation des W3C ist frei zugänglich und kompatibel zur Vorgängerversion OWL1 \cite{OWL1}, sie steht damit einer breiten Basis von Anwendungen und Anwendern zur Verfügung. Allerdings sind konkrete Umsetzungen von Software für OWL2 auf Grund seiner aktuellen Natur noch nicht vorhanden bzw. im Moment in der Entwicklung. In dieser Arbeit wird daher auf teilweise noch experimenteller Software, wie z.B. der OWLAPI v3 \cite{OWLAPI}, aufgesetzt. Dies ist nötig, um für die Zukunft eine aktuelle Schnittstelle für Ontologien zur Verfügung zu stellen.

Einige Prinzipien wie ein solcher Reasoner implementiert werden kann basieren dabei in einigen Teilen auf der Diplomarbeit "Speicherung und Abfrage großer Ontologien in deduktiven Datenbanken" aus dem Oktober 2003 von Timo Weithöhner. Sie ist damit auch eine Fortsetzung der Entwicklung des U2R2 (Uni Ulm Relational Reasoner). Es wird daher angenommen, das der Leser zumindest mit den Erkenntnissen und Ergebnissen dieser Arbeit vertraut ist.

Es wird hier im folgenden ein relationaler Reasoner mit forward-chaining und direct materialisation entwickelt. Relational steht dabei dafür, das die Ergebnisse und Fakten der Ontologien in Relationen in einer Datenbank abgespeichert werden. Forward-chaining bedeutet das aus existierenden Fakten durch Anwendung von Schlussfolgerungsregeln sofort neue Fakten erzeugen werden und darauf wieder die Schlussfolgerungsregeln angewendet werden. Dies geschieht bis keine neuen Fakten mehr erzeugt werden könnne. Direct-materialisation steht dafür, das alle erzeugten Fakten abgespeichert werden und damit zur Anfragezeit alle möglichen Antworten schon vorliegen und müssen nur dann nur noch ausgegeben werden.

Durch diese Eigenschaften ist es einerseits möglich große Ontologien zu laden und sie nicht vollständig im Arbeitsspeicher halten zu müssen, sondern können in Sekundärspeicher durch ein Datenbanksystem verwaltet sein. Andererseits wird durch den \emph{direct - materialisation} Ansatz der Fokus auf die Beantwortung von Fragen an die Ontologie gelenkt. Durch eine Vorberechnung ist es sehr schnell möglich darauf zu antworten.

Damit sollte die grundlegende Arbeitsweise schon aufgezeigt sein und der Leser sollte wissen was ihn erwartet.