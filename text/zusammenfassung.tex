\section*{Zusammenfassung}
Diese Arbeit beschäftigt sich mit der Implementierung und Optimierung eines relationalen Schlussfolgerers. Dieser Schlussfolgerer arbeitet dabei auf dem OWL2 RL Profil, das Ende Oktober 2009 vom W3C \cite{OWL2Profiles} veröffentlicht wurde. Dieser \emph{technical report} stellt eine Spezifikation der Semantik und der möglichen Schlussfolgerungsregeln in einer Ontologie dar. Die Veröffentlichung der Spezifikation des W3C ist frei zugänglich und kompatibel zur Vorgängerversion OWL1 \cite{OWL1}, sie steht damit einer breiten Basis von Anwendungen und Anwendern zur Verfügung. Allerdings sind konkrete Umsetzungen von Software für OWL2 auf Grund seiner aktuellen Natur noch nicht vorhanden bzw. im Moment in der Entwicklung. In dieser Arbeit wird daher auf teilweise noch experimenteller Software, wie z.B. der OWLAPI v3 \cite{OWLAPI}, aufgesetzt. Dies ist nötig, um für die Zukunft eine aktuelle Schnittstelle für Ontologien zur Verfügung zu stellen.

Einige Prinzipien, wie ein solcher Reasoner implementiert werden kann, basieren dabei in einigen Teilen auf der Diplomarbeit \emph{Speicherung und Abfrage großer Ontologien in deduktiven Datenbanken} aus dem Oktober 2003 von Timo Weithöhner. Sie ist damit auch eine Fortsetzung der Entwicklung des U2R2 (Uni Ulm Relational Reasoner). Es wird daher angenommen, das der Leser zumindest mit den Erkenntnissen und Ergebnissen dieser Arbeit vertraut ist.

Es wird im folgenden ein relationaler Reasoner mit \emph{forward-chaining} und \emph{direct materialisation} entwickelt. Relational steht dabei dafür, das die Ergebnisse und Fakten der Ontologien in Relationen in einer Datenbank abgespeichert werden. Forward-chaining bedeutet, dass aus existierenden Fakten durch Anwendung von Schlussfolgerungsregeln neue Fakten erzeugen werden und darauf wieder die Schlussfolgerungsregeln angewendet werden. Dies geschieht bis keine neuen Fakten mehr erzeugt werden können. Direct-materialisation steht dafür, dass alle erzeugten Fakten abgespeichert werden und somit zur Anfragezeit alle möglichen Antworten schon vorliegen und dann nur noch ausgegeben werden müssen.

Durch diese Eigenschaften ist es einerseits möglich große Ontologien zu laden und sie nicht vollständig im Arbeitsspeicher halten zu müssen, indem man sie im Sekundärspeicher durch ein Datenbanksystem verwaltet. Andererseits wird durch den direct-materialisation Ansatz der Fokus auf die Beantwortung von Fragen an die Ontologie gelenkt. Durch eine Vorberechnung ist es sehr schnell möglich darauf zu antworten. Das ist auch der angenommene Anwendungsfall für u2r3. Die Anzahl der Anfragen überwiegt bei weitem  die Zeit für die Vorberechnung. Die Zeit für die Beantwortung von Anfragen ist wichtiger niedrig zu halten, als die Zeit für die Vorberechnung.

Damit sollte die grundlegende Arbeitsweise aufgezeigt sein und der Leser sollte einen Überblick zum Gebiet der Implementierung haben.