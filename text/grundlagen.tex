\chapter{Grundlagen}
In den einleitenden Abschnitten wurden bereits einige Begriffe verwendet, die noch nicht zur Gänze erklärt sind bzw. die für das Verständnis der Entwicklungsgeschichte und auch für den technischen Gesichtspunkt wichtig sind.

Dazu gehören Ontologien, für was sie stehen und wozu sie nützlich sind. Die Beschreibung des OWL2 RL Fragments. Das Verständnis für einen Schlussfolgerer, sowie einige Hintergründen für die Umsetzungen von Ontologien in Datenbanken.

\section{Ontologien}

\section{OWL2}

\section{Schlussfolger}

\section{U2R2}

\begin{verbatim}
    * Überblick: Wie sieht der aktuelle Stand auf dem Onto-Gebiet/Semantic Web aus?
    * Wieso sind Ontos wichtig?
    * Was ist der Unterschied zwischen Ontologien und Datenbanken? 
    
    * Was ist OWL im Hinblick auf Ontologien?
         o Wo ist das OWL2 RL Fragment einzuordnen?
          o Wie ist OWL2 aufgebaut?
          o Was sind andere Profile
          o Was sind ihre Unterschiede
          o Was sind ihre Eigenschaften.
    * Was ist interessant bzgl des Fragements OWL2 RL?
    * Was ist das Einsatzgebiet von OWL2 RL?
    
    * Wie ist der OWL2 Zeitplan?
    
    * Warum braucht man einen Reasoner?
    * Was zeichnet einen Reasoner aus? Welche Funktionen muss er bereitstellen?
    
    * Was ist die Herkunft und der Hintergrund meiner DA?
    * Wie ordnet sich U2R2 ein? Was ist der Reasoner Ansatz/ die Reasoner Strategie?
    * Wie war dieser Reasoner aufgebaut und konzipiert?
    * Was waren die Fähigkeiten von U2R2?
    * Was ist das MEMA-Prinzip?
    
    
\end{verbatim}
