\chapter{Realisierung}

\section{Regelumsetzung von OWL2 RL}

In der Profil-Spezifikation von OWL2 RL \cite{OWL2Profiles} wird eine Reihe von Regeln angegeben, die es ermöglichen entsprechend der Sprachmächtigkeit darin zu schlussfolgern.
%\note Was ist die Sprachmächtigkeit?
%\note Was heißt schlussfolgern?
Die angegebenen Regeln sind vollständig aber nicht optimal \cite{placeholder}, d.h. mit diesem Satz an Regeln kann auf alles geschloßen werden, auf was im RL Profil geschloßen werden kann. Allerdings muss dieser Satz an Regeln der einzige oder minimale Satz sein, die zu diesem Ergebniss kommen..

In dieser Diplomarbeit wird versucht möglichst nah an diesem Menge von Regeln zu bleiben, um sichergestellt zu haben das die Semantik korrekt erhalten ist und somit die Implementierung leichter zu überprüfen ist.
Abweichungen von den vorgegebenen Regeln findet nur in besonderen Ausnahmen statt und werden später \ref{mitternachtsformel} auch einzeln begründet. Eine allgemeine Begründung für die Existenz der Ausnahmen liegt darin, dass die Regeln in SQL übersetzt werden - was schon von alleine eine andere Schreib- und Sichtweise  mitbringt - und zum anderen sind gewisse Optimierungen erwünscht.
Wieso und wie das funktioniert wird in Abschnitt \cite{placeholder} an einigen Beispielen und Sonderfällen demonstriert.

\subsection{Regeloptimierung}

Die Regelanwendungen sind der teuerste Teil im Programmablauf, darum gibt es hier verschiedene Möglichkeiten diesen Prozess zu beeinflussen und damit auf spezielle Bedürfnisse anpassen zu können.

Man kann entscheiden, ob man die Ontologie später verändern will. Das hat zur Folge, das die Entstehungsgeschichte von abgeleiteten Fakten erhalten bleiben muss.

Eine naive Anwendung der Regel hätte außerdem zur Folge, das sich die Regeln immer auf alle Fakten beziehen. Meist reicht es aber asich auf erzeugt Fakten aus vorherigen Schritten zu beziehen. Das wird näher im Abschnittt \cite{placeholder} (delta-Iteration) geklärt. Hierbei gibt es zwei verschiedene Varianten wie diese umgesetzt ist.

Die Struktur der Umsetzung der Regeln in SQL ist zwar fix aber durch gewisse Optimierungen abhängig vom Ausführungskontext. Zum einen wurden zwei verschiedene Varianten der delta-Iteration implmentiert und zum anderen müssen die Regeln, wenn ein kaskadierender Löschmodus aktiv ist auch Informationen mitführen, welche Fakten das aktuelle Faktum erzeugt haben.

\subsubsection{delta-Iteration}
\section{Delta-Iteration}
\label{abschnitt-delta-iteration}

Unabhängig von der Art der delta-Iteration muss immer der Ausführungskontext bekannt sein, d.h. welche delta-Iteration ist aktiv, auf welches delta wird gearbeitet. Ein delta muss dabei die Information enthalten auf welcher Relation es arbeitet und auf welchen Teil. Zum Ausführungskontext gehört je nach Modus auch noch die Information, ob eine Regel schon auf das Ziel delta angewendet wurde.

Die delta-Iteration ist ein Verfahren, um bei Regelanwendungen, die Datenmenge so zu reduzieren, das nur solche enthalten sind, die noch nicht berücksichtigt wurden und somit die tatsächlich zu betrachtenden Menge zu reduzieren.

Während der ersten Entwicklung sind dabei zwei Varianten der delta-Iteration klar geworden. Diese tragen im folgenden die Namen \emph{immediate} delta-Iteration und \emph{collective} delta-Iteration. Timo Weithöhner hat in seiner Implementierung die collective delta-Iteration verwendet.

\subsection{Immediate delta-Iteration}

Hierbei erzeugt jede Regelanwendung sofort ihr eigenes Delta, mit dem weitergearbeitet werden kann. Wenn Zeilen erzeugt wurden, wird sofort das dazugehörige Delta erzeugt und dafür alle notwendigen Regelanwendungen ausgelöst.

Im folgenden ist ein Beispiel gegeben, wie der Ablauf bei der Anwendung einer Regel ist die neue Fakten erzeugt hat und wie diese dann verarbeitet wird. Angenommen die Regel \emph{eq-ref-s} erzeugt neue Fakten, diese landen in einem Delta der \emph{sameas}-Relation. Das löst eine ``Reason'' aus mit der Information welches Delta neue Fakten enthalten hat. Wenn der Regelprozessor diese erhält erzeugt er für diese Kombination sog. RuleActions, d.h. er bringt alle Regeln die auf diesem delta arbeiten in eine Warteschlange. Zu jeder Regel speichert er den Ausführungskontext, auf den sie angewendet werden soll mit ab. Die Regelanwendungen warten dann bis sie zur Ausführung gebracht werden.

\begin{verbatim}
Regel: eq-ref-s //neue Fakten
=> delta sameas
   => new Reason(sameas, delta)
      => new RuleAction(eq-sym, delta)
         new RuleAction(eq-trans, delta)
         new RuleAction(eq-rep-s, delta)
         new RuleAction(eq-rep-p, delta)
         new RuleAction(eq-rep-o, delta)
         new RuleAction(eq-diff1, delta)
         new RuleAction(eq-diff2, delta)
         new RuleAction(eq-diff3, delta)
\end{verbatim}

Der Regelprozessor ist dann auch verantwortlich die Ausführung der Regelanwendungen zu starten. Das passiert in einer Schleife, die die Elemente der Warteschlange abarbeitet bis keine weiteren Elemente vorhanden sind.

Dies ist im folgenden Code \ref{code-immediate-delta-iteration} schematisch dargestellt:

\begin{figure}[htp]
	\caption{Abarbeitung der RuleActions im immediate Modus.}
	\label{code-immediate-delta-iteration}
	\begin{lstlisting}[language=Java]
while(action = rulesToApply.popAction()) {
	action.rule.apply(action.delta);
	
	if (!(rulesToApply.contains(delta)) {
		delta.addToRelation();
		delta.drop();
		delta == null;
	}
}
	\end{lstlisting}
\end{figure}


Der Vorteil dieser Vorgehensweise ist das es besser parallelisiert werden kann, da soviele Actions wie möglich gleizeitig angewand werden können. Die Parallelisierung ist eines der Ziele die eventuell in der Zukunft umgesetzt werden sollen.

Der Nachteil ist das viele Deltas gleichzeitg entstehen und insgesamt mehr Regel angewendet werden müssen. Die Deltas enthalten unter Umständen nur wenig Inhalt, d.h. die Abfragen arbeiten auf kleineren Datenmenge aber ein Datenbanksystem lohnt sich eventuell erst wirklich bei großen Datenmengen. Es muss nicht zwangsläufig schlecht sein, aber es ist nicht unbedingt klar wie sich das auf ein DBMS auswirkt.

\subsection{Collective delta-Iteration}

Vom aktuellen Stand einer Relation werden erst einmal alle Änderungen in einer Hilfsrelation gesammelt. Sind alle abgearbeitet, wird das Delta erstellt und die nächste Phase beginnt.

\begin{verbatim}
Regel: sameas-eq-ref-s //neue Fakten
=> aux sameas
\end{verbatim}

\begin{figure}[htp]
	\caption{Abarbeitung der RuleActions im immediate Modus.}
	\label{code-immediate-delta-iteration}
	\begin{lstlisting}[language=Java]
do {
	while (action = list.popAction()) {
		action.rule.apply(action.delta);
	}
}
while (applyRelations()); //WAIT --- SYNC

boolean applyRelations() {
	for(r : relations)
		r.applyAux();
		return relations.wereDirty()
}

applyAux() {
	createDelta;
	addDelta();
	clearAux();
	new Reason(relation, delta);
}
	\end{lstlisting}
\end{figure}

Vorteile:
\begin{itemize}
  \item Es gibt immer nur eine Hilfsrelation und ein Delta, also eine fixe Anzahl an Tabellen.
  \item Es werden evtl. größere Menge an neuen Fakten zusammengefasst. Das muss nicht notwendiger weise gut sein.
\end{itemize}

Ein Nachteil ist, das die Parallelsierung nicht vollständig durchgeführt werden kann, da zu gewissen Zeitpunkten (Zeile 6), die Regelanwendung unterbrochen wird und die Ausführung wieder synchronisiert werden muss.

\subsection{Simulation}

Die collective delta-Iteration kann mit Hilfe der immediate delta-Iteration und ein paar Änderungen umgesetzt werden. Dazu dürfen die Relationen nicht sofort neue Hinweise an den Regelprozessor schicken, sondern es gibt eine eigene Phase nach dem Abarbeiten aller Regeln. Dabei wird erst der Inhalt der Deltas in die Haupttabelle übernommen und es werden die Reasons an den Regelprozessor geschickt.

Die Hilfstabelle in der die Daten für die Runde zwischengespeichert werden ist das Delta, das neu angelegt wirde. Die alte Hilfstabelle wird zum Delta der aktuellen Runde.

Die Implementierung des u2r3 unterstützt beide Varianten der delta-Iteration. Diese können in der Konfigurationsdatei mit der Option DeltaIteration verändert werden. Gültige Werte sind \emph{COLLECTIVE} oder \emph{IMMEDIATE}.


\subsubsection{Magic-Sets}
Auf die Implementierung von Magic-Set-Algorithmen kann verzichtet werden, da die Regel-Implementierung in SQL fast ausschließlich von Hand abliefund deswegen schon von vorneherein optimiert sind.
Referenzen-MEHR.


\subsection{Regelbeispiele}
\subsection{Regelbeispiele}
In diesem Teil der Regelumsetzung wird an konkreten Beispielen gezeigt, wie eine Umsetzung der OWL2 RL Regeln in SQL möglich ist. Die SQL Abfragen benutzen dabei die vorher im MEMA-Prinzip erstellen Relationen

Die Abfragen werden zunächst allgemein an einfachen Regeln prinzipiell erklärt, danach wird auf die Sonderfälle eingegangen.

Insgesamt kann dabei in vier Kategorie unterschieden werden:
\begin{itemize}
  \item Gewöhnliche Regeln: Dabei werden verschiedene Fakten miteinander verknüpft, so das dabei neue Fakten entstehen können. Diese Regeln sind alle sehr schematisch umsetzbar.
  \item Listenregeln: Hier werden Listen von Fakten bearbeitet. Dabei ist nicht klar wie viele Elemente eine Liste enthält. Dies kann zu einem Problem werden, wenn man die Entstehungsgeschichte von Fakten, z.B. für das Löschen mitabspeichern will. Hier muss man gesondert darauf achten das diese Information der Entstehung nicht verloren geht.
  \item Inkonsistenzregeln: Sie sind ähnlich der neuen Regeln, erzeugen allerdings keine neuen Fakten. Falls sie neue Fakten erzeugen könnten bedeutet dies eine Inkonsistenz in der Ontologie.
  \item Regeln für Datentypen: Für typisierte Literal werden einige Überprüfung bzgl. der Gleichheit untereindander und der Konformität zu den in OWL2 RL eingebauten Datentypen durchgeführt.
  \item Einmalige Regeln: Einige Regeln haben keine Vorbedigung. Diese werden einmal zu Beginn der Regelanwendung ausgeführt.
\end{itemize}

\subsubsection{Einmalige Regeln}

cls-thing
if | then
true | owl:Thing rdf:type owl:Class

Diese Regel wird einmal zu Beginn in die Liste der Regelanwendungen gesteckt. Sie sorgt dafür das dieses Faktum eingefügt wird.

In SQL lautet dies:

INSERT INTO classAssertionEnt (entity, class)
	VALUES ('owl:Thing', 'owl:Class')

\subsubsection{Inkonsistenzregeln}
Inkonsistenzregeln erzeugen keine Fakten. Sie versuchen aber gewisse Fakten zu finden, die im Widerspruch zu einander stehen.

Die Inkonsistenzregeln cls-nothing2 sieht dabei so aus:

If | then
T(?x, rdf:type, owl:Nothing) | false

Dies wird in folgende SQL-Abfrage umgewandelt:

SELECT 1
FROM classAssertionEnt
WHERE class = 'owl:Nothing'

Die Relation classAssertionEnt ist die Relation, die die type-Beziehung speichert. Ist darin eine Zeile zu finden, die als type die KLasse owl:Nothing hat wird eine Zeile zurückgegeben. Falls also diese Abfrage eine oder mehrere Zeilen erzeugt liegt eine Inkonsistenz vor.

\subsubsection{Gewöhnliche Regeln}
Gewöhnliche Regeln erzeugen neue Fakten in der Datenbank. Eine einfache Regel ist hier eq-sym:

If | then
T(?x, owl:sameAS, ?y) | T(?y, owl:sameAS, ?x)

Dies wird wie folgt in SQL überführt:

INSERT INTO sameAs (left, right)
SELECT right, left
FROM sameAs

Wie die Abfrage zu Stande kommt sollte klar sein. Allerdings wurden in dieser Umwandlung schon einge Dinge vereinfacht, die in der Implementierung so nicht gemacht wurden.

\begin{enumerate}
  \item Was passiert wenn eine Zeile eingefügt werden soll, die schon enthalten ist?
  \item Wie würde man hier die Delta-Iteration einsetzen können, um nicht immer auf alle Fakten schließen zu müssen?
  \item Wie kann man die Entstehungsgeschiechte von neuen Fakten mitschreiben, um später effizientes Löschen zu ermöglichen.
\end{enumerate}

Diese Punkte werden jeweils in ihren speziellen Abgeschnitten. Um die Beispiel nicht untnötig komplizierter zu machen wird hier nicht näher darauf eingegangen.

Ein komplexeres Beispiel ist die Verknüpfung von Fakten um auf neue Fakten schließen zu können, z.B. in der Regel eq-rep-s:

If | then
T(?s1, owl:sameAs, ?s2), T(?s1, ?p, ?o) | T(?s2, ?p, ?o)

Die Verknüpfung wird durch einen JOIN auf die entsprechende Spalte realisiert.

INSERT INTO objectPropertyAssertion(subject, object, property)
SELECT sa.right, opa.property, opa.object
FROM objectPropertyAssertion AS opa
INNER JOIN sameAs AS sa
	ON sa.left = opa.subject


\section{Abfragen}
\subsection{Komplexe Ausdrücke finden}
Beim Suchen von komplexen Ausdrücken hat man es damit zu tun, das der Ausdruck zur Speicherung in den Relationen aufgeteilt werden musste. Diese Aufteilung ist mit von sog. NodeId passiert, wie sie in der OWLAPI auch für anonyme Individuen verwendet wird, werden sie auch hier verwendet um eine Beziehung zwischen Teilen herzustellen.. Was aber der Wert dieser IDs war ist bei einer späteren Anfrage nicht mehr bekannt bzw es ist auch nicht bekannt wie diese auf den komplexen Ausdruck passen sollten oder obder Ausdruck überhaupt in der geschlussfolgerten Menge liegt

Um den Ausdruck nicht in die Datenbank einfügen zu müssen und ihn durch Schlussfolgern gleichsetzen zu können wird dieser Ausdruck rekursiv durchlaufen und darus ein SQL-Ausruck aufgebaut. Dieser SQL-AUsdruck kann den Namen der Relation, sowie die ID des Konstrukts zurückliefern. Beim AUfbau der SQL-Abfrage erzeugt jedes Unterkonstrukt des komplexen Ausdrucks eine SQL-Unterabfrage. Dabei ist zu beachten, das die Unterabfragen eine Referenz zu ihrer ``Oberabfrage'' enthält.

Nachfolgend ein Beispiel, dabei wird ein komplexer OWL-Ausdruck\footnote{Der Ausdruck wird in der FunctionalSyntax dargestellt. Damit ist die Hierarchie so wie das rekursive Vorgehen leicheter nachvollziehbar.} in eine SQL-Abfrage umgewandelt.

\begin{verbatim}
ClassAssertion(ind1, SomeValuesFrom(prop1, cls1))
\end{verbatim}

\begin{lstlisting}[language=SQL]
SELECT id, 'classAssertion'
FROM classAssertion AS t1
WHERE entity = 'ind1' AND
	EXISTS (
	SELECT id, 'someValuesFrom'
	FROM someValuesFrom AS t2
	WHERE t2.class = t1.class AND
		property = 'prop1' AND
		total = 'cls1'
	)
\end{lstlisting}


Beim Löschen werden ebenfalls möglicherweise komplexe Ausdrücke angegeben. Um diese aufzufinden können die selben Methoden verwendet werden.

\section{Löschung}
Modusname: '''!DeletionType''' := CLEAN | CASCADING

== Ableitungsreihenfolge abspeichern ==
 * OWL2 RL ist eine monotone Logik ([ticket:20])
   * Neue Fakten können einfach hinzugefügt werden und darüber erneut geschloßen werden. Alte Fakten werden dadurch nicht beeinflusst.
   * Änderungen von Daten können auf löschen von Fakten und hinzufügen von Fakten reduziert werden. Hinzufügen ist einfach (siehe oben).
   * Löschen von Fakten
     * Fakten die abgeleitet wurden können nicht gelöscht werden
     * Fakten die keine Ableitungen verursacht haben können gelöscht werden
     * Fakten die eine Ableitung verursacht haben können nur gelöscht werden, wenn alle ihre Ableitungen gelöscht werden.

Durch das Speichern der Ableitungsreihenfolge Manipulationen an der Ontologie beschleunigen.

=== Woher ist etwas abgeleitet ===
 * eine Regel
 * '''mehrere Fakten'''
Regel egal, da diese ja auf Grund der Fakten ausgewählt wurde.
''Gilt das nur für die ABox oder auch die TBox?'' '''ja''

=== Wie speichert man die Ableitungsreihenfolge ab ===
Die !ApplicationRules so erweitern, dass sie von jeder verwendeten Tabelle, die id der Zeile mitliefert. Dazu müssen ids in Tabellen eingeführt werden. Das entspricht sehr dem !MEMA-Schema, da durch id eine klare, einheitlich Historie-Tabelle angelegt werden kann, die unabhängig von unterschiedlichen Axiom-Konstrukten ist. Die ids müssen !UNIQUE pro Tabelle sein.

Falls die Anzahl der Spalten einer Abfrage nicht bekannt ist, kann sie auch dynamisch erstellt werden, da die einheitliche ids pro Axiom verwendet werden. Die Anzahl der Spalten pro Tabelle ist von H2 nicht begrenzt (schon überprüft, in der Dokumentation).

Danach müssen die Daten-Spalten in eine seperate Referenztabelle (Historien-Tabelle). Hier wird zu einer id und einer Tabelle, die Quelldaten woher dieser Fakt stammt ebenfalls in Form von einer id und Tabelle abgepseichert.

Regel
%{{{
subClass(A,C) := subClass(A,B), subClass(B,C)
 neue\_id             id\_x           id\_y
%}}}

Historie
%{{{
   id      ||   quelle
========================
 neue\_id         id\_x
 neue\_id         id\_y 
%}}}


'''ABER'''

Den Deltas kann im voraus keine deterministische id gegeben werden oder nur mit erheblichen Locking- und Management-Aufwand. Daher liegt es nahe sog. !UUIDs zu verwenden. Das ist allerdings ein kleiner Performance- und Speicherverlust. Verliert aber dabei nicht an Skalierbarkeit.

==== Idee ====
Evtl. können auch Daten mit einer Art Wildcard angelegt werden. Daten, die keine Quellangabe haben werden immer gelöscht.

=== Wie löscht man dann Daten ===
Die id des zu löschenden Axioms bestimmen.
Schauen, ob die id in der Historientabelle vorhanden ist. Wenn ja dann alle Abhängigkeiten finden. Rekursiv diese löschen. Dann den Wert selbst löschen.

z.B.: subClass(B,C) => id\_y
%{{{
SELECT id FROM Historie WHERE quelle = $id_y$

while(id)
  lösche(id)
  DELETE FROM Historie WHERE id = id

DELETE FROM subClass WHERE id = $id_y$
%}}}

=== Kann man Unterschiede in ABox und TBox Löschungen ausnützen? ===
In U2R2/3 verschwimmt der Unterschied zwischen ABox und TBox weiter, da beide in den selben Strukturen abgespeichert werden.


== Wann müssen beim Löschen Regeln ausglöst werden? ==
Es müssen für alle gelöschte Fakten Reason augelöst werden, da nicht von vornherein gesagt werden kann, wie Daten erzeugt werden.

== Welche Regeln müssen beim Löschen ausgelöst werden? ==
Relationen müssen wissen welche Regeln bei Ihnen Daten erzeugen können. Diese Regeln müssen dann ausgelöst werden. Im Gegenteil zu ''normalen'' Inferenzregeln.

Beim löschen müssen die Regeln angestoßen werden, die in der gelöschten Tabelle Daten erzeugen könnten. Für die Unterscheidung sollten zwei verschiedene Reason eingeführt werden.
 * !AdditionReason (wenn der Grund durch das hinzufügen eines Faktes entstanden ist)
 * !DeletionReason (wenn der Grund durch das hinzufügen eines Faktes entstanden ist)

\section{Konsistenzprüfungen}
Um die Konsistenz einer Ontologie aus dem OWL2 RL Profil sicherzustellen, sind in der Spezifikation eigene Regeln vorgegeben. Es ist dabei aber nicht vorgeschrieben wann diese Regeln zur Üpberprüfung zur Ausführung gebracht werden müssen. Außerdem muss eine Ontologie unter Umständen nicht immer auf ihre Konsistenz überpüft werden, daher ergeben sich zunächst folgende Ideen wann und wie man die Prüfungen durchführt.

\begin{itemize}
  \item Nach jeder Regelanwendung: Dies ist nicht sinnvoll, da durch eine Regelanwendung nicht gegen beliebige Konsistenzregeln verstoßen werden kann.
  \item Nur abhängige Konsistenzregeln überprüfen: Die Auslösung der Regeln funktioniert dabei genauso wie bei gewöhnlichen Regeln. Dieses Verfahren meldet sofort zurück wenn eine Inkonsistenz gefunden wurde, macht aber nur den nötigsten Aufwand. Es ist in u2r3 implementiert und ist der voreingestellt Modus zur Konsistenzprüfung.
  \item Die Überprüfung könnte erst nach kompletter Ableitung durchgeführt werden. Es ist allerdings fraglich, ob es soviel schneller ist, als nach nur abhängigen. Außerdem kann auch schon eine halbabgeleitete Ontologie nützlich sein, man wüsste aber dann nicht, ob diese konsistent ist.
  \item Keine Überprüfung der Konsistenz ist sicherlich die schnellste Möglichkeit, da hierbei einfach keine Regeln ausgeführt werden müssen. Dies ist hilfreich, wenn man z.B. die Konsistenz schon in einem vorherigen Lauf bestimmt hat. Diese Option ist ebenfalls implementiert.
\end{itemize} 
Die Möglichkeit die Konsistenzprüfung zu wechseln ist in u2r3 implementiert und kann in der Konfigurationsdatei unter dem Namen \emph{ConsistencyLevel} gesetzt werden. Gültige Werte dabei sind \emph{NONE} oder \emph{DEFAULT}.

Damit der Regelprozessor zwischen Konsistenzregeln und Anwendungsregeln unterscheiden kann sind alle Regeln die eine Inkonsistenz auslösen könne von der Subklasse ConsistencyRule abgeleitet.

\subsection{Rückmeldung}
Konsistenzregeln melden sich nur in einem Fehlerfall zurück. Die Rückmeldung erfolgt aus den Regeln selbst während ihrer Ausführung. Die OWLAPI schlägt hier vor eine Exception zu werfen. Allerdings sind nicht alle Inkonsistenz gleich schwerwiegend. Eine Inkonsistenz in der A-Box ist normalerweise nicht so problematisch wie eine Inkonsistenz in der T-Box.

Die Möglichkeiten wie auf eine Inkonsistenz reagiert werden kann sind dabei:
\begin{itemize}
  \item Nur eine Warnung auszugeben, diese wird dann automatisch mitgeloggt. Das ist die Voreinstellung. Damit wird der Schlussfolgerungsprozess nie unterbrochen.
  \item Abhängig von der Inkonsistenz reagieren. Dabei wird zwischen Inkonsistenzen in der A-Box und der T-Box unterschieden. Ein Fehler in der A-Box gibt nur eine Warnung aus, wohingegen ein Fehler in der T-Box eine Ausnahme erzeugt.
  \item Die Verarbeitung unterbrechen und bei jeder Inkonsistenz eine Exception erzeugen.
\end{itemize}

Im Reasoner kann dies mit der Option \emph{InconsitencyReaction} verändert werden, dabei sind \emph{WARN}, \emph{PERCASE} oder \emph{FAIL}.



