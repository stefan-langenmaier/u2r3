\chapter{Realisierung}

\section{Regelumsetzung allgemein}
\label{abschnitt-regelumsetzung}
Regeln sind hier sogenannte Inferenzprozeduren, das sind automatisierte Verfahren zur Berechnung logischer Folgerungen. Inferenzprozeduren sind bereits aus der Aussagenlogik bekannt.

\begin{table}[htb]
\begin{center}
	\begin{tabular}{l}
	A$\rightarrow$B \\
	A \\
	\hline
	B
	\end{tabular}
\end{center}
	\caption{Modus ponens}
	\label{table-modus-ponens}
\end{table}

\begin{table}[htb]
\begin{center}
	\begin{tabular}{l}
	A$\rightarrow$B \\
	$\neg$B \\
	\hline
	$\neg$A
	\end{tabular}
\end{center}
	\caption{Modus tollens}
	\label{table-modus-tollens}
\end{table}

Diese beiden Ableitungsregeln [\ref{table-modus-ponens}, \ref{table-modus-tollens}] erlauben es, die beiden Aussagen in der Prämisse durch die Schlussfolgerung zu ersetzen. Für einen Schlussfolgerer mit direct-materialization werden, die Prämissen nicht ersetzt sondern weiterhin erhalten. Es könnte ja möglich sein das weitere Ableitungsschritte damit möglich sind oder das spätere Anfragen auf diese Aussagen abzielen. Mit einer Ableitungsregel kann also die Aussagenbasis vergrößert werden.

Es sind dabei jedoch nicht alle Ableitungsregeln gültig.
\begin{table}[htb]
\begin{center}
	\begin{tabular}{l}
	A$\rightarrow$B \\
	B \\
	\hline
	A
	\end{tabular}
\end{center}
	\caption{ungültige Ableitung}
	\label{table-invalid-inferred}
\end{table}
Abbildung \ref{table-invalid-inferred} ist z.B. kein gültiger Schluss.

Eine Menge von vernünftigen Ableitungsregeln zu finden ist keine leichte Aufgabe. Dabei eine möglichst große Ausdrucksmächtigkeit zu erhalten, ohne die Komplexität der Regeln aus den Augen zu verlieren wurde versucht mit dem OWL2 RL Fragement zu erreichen.

\subsection{Regelbeispiele}
\label{abschnitt-regelbeispiele}
In diesem Teil der Regelumsetzung wird an konkreten Beispielen gezeigt, wie eine Umsetzung der OWL2 RL Regeln in SQL möglich ist. Die SQL Abfragen benutzen dabei die vorher im MEMA-Prinzip erstellen Relationen

Die Abfragen werden zunächst allgemein an einfachen Regeln prinzipiell erklärt, danach wird auf die Sonderfälle eingegangen.

Insgesamt kann dabei in fünf Kategorien unterschieden werden:
\begin{itemize}
  \item Gewöhnliche Regeln: Dabei werden verschiedene Fakten miteinander verknüpft, so dass dabei neue Fakten entstehen können. Diese Regeln sind alle sehr schematisch umsetzbar.
  \item Listenregeln: Hier werden Listen von Fakten bearbeitet. Dabei ist nicht klar wie viele Elemente eine Liste enthält. Dies kann zu einem Problem werden, wenn man die Entstehungsgeschichte von Fakten, z.B. für das Löschen mitabspeichern will. Hier muss man gesondert darauf achten, dass diese Information der Entstehung nicht verloren geht.
  \item Inkonsistenzregeln: Sie sind ähnlich der gewöhnlichen Regeln, erzeugen allerdings keine neuen Fakten. Falls sie neue Fakten erzeugen könnten, bedeutet dies eine Inkonsistenz in der Ontologie.
  \item Regeln für Datentypen: Für typisierte Literale werden einige Überprüfung bzgl. der Gleichheit untereinander und der Konformität zu den in OWL2 RL eingebauten Datentypen durchgeführt.
  \item Einmalige Regeln: Einige Regeln haben keine Vorbedigung. Diese werden einmal zum Start des Regelprozessors ausgeführt.
\end{itemize}

\subsubsection{Einmalige Regeln}

\begin{table}[htb]
\begin{center}
	\begin{tabular}{l|l}
	if & then \\ \hline
	true & T(owl:Thing, rdf:type, owl:Class)
	\end{tabular}
\end{center}
	\caption{Die Regel cls-thing}
	\label{rule-cls-thing}
\end{table}


Diese Regel \ref{rule-cls-thing} wird einmal zu Beginn in die Liste der Regelanwendungen gesteckt. Sie sorgt dafür das dieses Faktum eingefügt wird.

In SQL lautet dies:
\begin{lstlisting}[language=SQL]
INSERT INTO classAssertionEnt (entity, class)
	VALUES ('owl:Thing', 'owl:Class')
\end{lstlisting}

\subsubsection{Inkonsistenzregeln}
Inkonsistenzregeln erzeugen keine Fakten. Sie versuchen aber gewisse Fakten zu finden, die im Widerspruch zu einander stehen.

\begin{table}[htb]
\begin{center}
	\begin{tabular}{l|l}
	if & then \\ \hline
	T(?x, rdf:type, owl:Nothing) & false
	\end{tabular}
\end{center}
	\caption{Die Inkonsistenzregel cls-nothing2}
	\label{rule-cls-nothing2}
\end{table}


Dies wird in folgende SQL-Abfrage umgewandelt:
\begin{lstlisting}[language=SQL]
SELECT 1
FROM classAssertionEnt
WHERE class = 'owl:Nothing'
\end{lstlisting}

Die Relation classAssertionEnt ist die Relation, die die type-Beziehung speichert. Ist darin eine Zeile zu finden, die als type die KLasse owl:Nothing hat wird eine Zeile zurückgegeben. Falls also diese Abfrage eine oder mehrere Zeilen erzeugt liegt eine Inkonsistenz vor.

\subsubsection{Gewöhnliche Regeln}
Gewöhnliche Regeln erzeugen neue Fakten in der Datenbank. Eine einfache Regel ist hier eq-sym:

\begin{table}[htb]
\begin{center}
	\begin{tabular}{l|l}
	if & then \\ \hline
	T(?x, owl:sameAS, ?y) & T(?y, owl:sameAS, ?x)
	\end{tabular}
\end{center}
	\caption{Die gewöhnliche Regel eq-sym}
	\label{rule-eq-sym}
\end{table}

Dies wird wie folgt in SQL überführt:
\begin{lstlisting}[language=SQL]
INSERT INTO sameAs (left, right)
SELECT right, left
FROM sameAs
\end{lstlisting}

Wie die Abfrage zu Stande kommt sollte klar sein. Allerdings wurden in dieser Umwandlung schon einge Dinge vereinfacht, die in der Implementierung so nicht gemacht wurden.

\begin{enumerate}
  \item Was passiert wenn eine Zeile eingefügt werden soll, die schon enthalten ist?
  \item Wie würde man hier die Delta-Iteration einsetzen können, um nicht immer auf alle Fakten schließen zu müssen?
  \item Wie kann man die Entstehungsgeschiechte von neuen Fakten mitschreiben, um später effizientes Löschen zu ermöglichen.
\end{enumerate}

Diese Punkte werden jeweils in ihren speziellen Abschnitten behandelt. Um die Beispiele nicht unnötig komplizierter zu machen wird hier nicht näher darauf eingegangen.

Ein komplexeres Beispiel ist die Verknüpfung von Fakten, um auf neue Fakten schließen zu können, z.B. in der Regel eq-rep-s:
\begin{table}[htb]
\begin{center}
	\begin{tabular}{m{4.5cm}|m{4cm}}
	if & then \\ \hline
	T(?s1, owl:sameAs, ?s2),\newline T(?s1, ?p, ?o) & T(?s2, ?p, ?o)
	\end{tabular}
\end{center}
	\caption{Die Regel eq-rep-s}
	\label{rule-eq-rep-s}
\end{table}

Die Verknüpfung wird durch einen JOIN auf die entsprechende Spalte realisiert.

\begin{lstlisting}[language=SQL]
INSERT INTO objectPropertyAssertion(subject, object, property)
SELECT sa.right, opa.property, opa.object
FROM objectPropertyAssertion AS opa
INNER JOIN sameAs AS sa
	ON sa.left = opa.subject
\end{lstlisting}

\subsubsection{Listenegeln}
Dabei können Listenregeln auch Regeln sein die neue Fakten erzeugen oder Inkonsistenzen überprüfen. Eine Inkonsistenzregel mit einer Liste ist z.B. prp-adp:
\begin{table}[htb]
\begin{center}
	\begin{tabular}{M{7cm}|M{1.2cm}|M{2.5cm}}
	if & then & \\ \hline
	T(?x, rdf:type, owl:AllDisjointProperties),\newline
	T(?x, owl:members, ?y),\newline
	LIST[?y, ?$p_1$, \ldots, ?$p_n$],\newline
	T(?u, ?$p_i$, ?v),\newline
	T(?u, ?$p_j$, ?v) & false & für alle $1 \leq i < j \leq n$
	\end{tabular}
\end{center}
	\caption{Die Listenregel prp-adp}
	\label{rule-prp-adp}
\end{table}


Bei der Erstellung der Abfrage kann man rein schematisch vorgehen. Es werden alle Relation auf den Variablen mit den gleichen Namen gejoint. Liefert dese Abfrage ein Ergebnis zurück ist eine Inkonsistenz vorhanden.

\begin{lstlisting}[language=SQL]
SELECT 1
FROM classAssertionEnt AS ca
INNER JOIN members AS m ON ca.entity = m.class
INNER JOIN list AS l ON m.list = l.name
INNER JOIN objectPropertyAssertion AS opa1
	ON opa1.property = l.property
INNER JOIN objectPropertyAssertion AS opa2
	ON opa2.property = l.property
WHERE opa1.subject = opa2.subject AND opa1.object = opa2.object
	AND ca.class = 'owl:AllDisjointProperties'
\end{lstlisting}

Tatsächlich wird hier mehr gemacht als nötig ist. Die \emph{subject} und \emph{object} Spalten werden von beiden Seiten miteinander verglichen. Es wäre aber nur eine nötig. So ist es allerdings einfacher und natürlicher in SQL Syntax zu implementieren.

Im Falle der Regel eq-diff2 muss eine Tabelle doppelt importiert werden, um wirkliche alle Fakten mit einander vergleichen zu können. Die ursprüngliche Regel.
\begin{table}[htb]
\begin{center}
	\begin{tabular}{M{7cm}|M{1.2cm}|M{2.5cm}}
	if & then & \\ \hline
	T(?x, rdf:type, owl:AllDifferent),\newline
	T(?x, owl:members, ?y),\newline
	LIST[?y, ?$z_1$, \ldots, ?$z_n$],\newline
	T(?$z_i$, owl:sameAs, ?$z_j$) & false & für alle $1 \leq i < j \leq n$
	\end{tabular}
\end{center}
	\caption{Die Listenregel eq-diff2}
	\label{rule-eq-diff2}
\end{table}


Würde man hier streng nach Schema vorgehen würde die sameAs Relation nur einmal in der Abfrage auftauchen. Das würde allerdings nicht alle Kombinationen erzeugen. Die Abfrage ist damit sehr ähnlich der obigen und wird hier ausgespart.

Ein besonderer Typ der Listenregel ist z.B. cls-oo, diese Regel erzeugt mehrere Fakten. Das ist aber in der Umsetzung kein Problem.
\begin{table}[htb]
\begin{center}
	\begin{tabular}{m{4.5cm}|m{4cm}}
	if & then \\ \hline
 	T(?c, owl:oneOf, ?x),\newline
 	LIST[?x, ?$y_1$, \ldots, ?$y_n$]
								 	&
								 	T(?$y_1$, rdf:type, ?c),\newline
								 	\ldots,\newline
								 	T(?$y_n$, rdf:type, ?c)
	\end{tabular}
\end{center}
	\caption{Die Listenregel cls-oo}
	\label{rule-cls-oo}
\end{table}


\subsubsection{Datentypenregeln}
Die Regeln zur Überprüfung der Datentypen ist nicht komplizierter, allerdings eine effiziente Umsetzung ist hier das Problem. Hier ist vor allem die dt-eq Regel das Problem.
\begin{table}[htb]
\begin{center}
	\begin{tabular}{l|l|M{3cm}}
	if & then & \\ \hline
	true & T(lt1, owl:sameAs, lt2) & für alle Literale lt1 und lt2 mit dem selben Datenwert
	\end{tabular}
\end{center}
	\caption{Die Datentyperegel dt-eq}
	\label{rule-dt-eq}
\end{table}


Die aktuelle Umsetzung versucht sich möglichst nah an das MEMA-Prinzip und die Regel zu halten. Optimierungen sind hier noch wesentlich vorhanden.

\begin{lstlisting}[language=SQL]
INSERT INTO sameAsLit (left, right)
SELECT ca1.literal, ca2.literal
FROM classAssertionLit AS ca1
CROSS JOIN classAssertionLit AS ca2
WHERE isSameLiteral(ca1.literal, ca2.literal)
\end{lstlisting}

Zum einen wird hier ein CROSS JOIN verwendet der eine der aufwendigsten Datenbankoperationen ist, außerdem können Literale nicht mit den üblichen Datenbankoperatoren verglichen werden. Hierzu wurde eine eigene Datenbankmethode geschrieben die dies übernimmt. Die Datenbankmethode ruft eine Java-Routine auf, die dann den eigentlichen Vergleich vornimmt. Diese beiden Verlangsamungen sind noch nicht genug. In der aktuellen IMplementierung sind nicht gleich alle Literal in der Relation, sondern können später noch hinzukommen, d.h. das diese Regel öfters aufgerufen werden kann. Hier ist also noch Potential für eine Optimierung vorhanden.


\subsubsection{Spezialfälle} 
In die Kategorie der Spezialfälle fallen drei Regeln aus OWL2 RL. Sie haben im Gegensatz zu den anderen Regeln eine variable Anzahl von Fakten in der Prämisse. Diese Fakten sind allerdings nur indirekt über eine Liste verknüpft.

Die erste Regel ist prp-spo2, sie erweitert eine Liste in eine Kette von Eigenschaften.
\begin{table}[htb]
\begin{center}
	\begin{tabular}{m{7cm}|m{3cm}}
	if & then \\ \hline
	T(?p, owl:propertyChainAxiom, ?x),\newline
	LIST[?x, ?$p_1$, \ldots, ?$p_n$],\newline
	T(?$u_1$, ?$p_1$, ?$u_2$),\newline
	T(?$u_2$, ?$p_2$ ?$u_3$),\newline
	\ldots,\newline
	T(?$u_n$, ?$p_n$, ?$u_{n+1}$) & T(?$u_1$, ?p, ?$u_{n+1}$)
	\end{tabular}
\end{center}
	\caption{Der Spezialfall prp-spo2}
	\label{rule-prp-spo2}
\end{table}

Die Kette muss dabei einen Start und ein Ende haben und die Glieder der Kette müssen untereinander verbunden sein. Die Variable ?u2 wird einmal als object und einmal als subject verwendet. Wenn das so ist ergeben der Start, die Eigenschaft und das Ende einen neues Faktum.

Diese Regel ist in SQL in zwei Schritten übersetzt worden. Zunächst eine Hilfsabfrage die in der späteren Abfrage mehrmals verwendet wird.

\begin{lstlisting}[language=SQL]
SELECT vopa.subject as vorgaenger,
       opa.subject  AS start,
       opa.object   AS ende,
       nopa.object  as nachfolger,
       l.name       AS lname,
       anzl.anz
FROM   list AS l
       INNER JOIN objectpropertyassertion AS opa
         ON opa.property = l.element
       INNER JOIN (SELECT   name,
                            COUNT(name) AS anz
                   FROM     list
                   GROUP BY name) AS anzl
         ON anzl.name = l.name
       LEFT OUTER JOIN objectpropertyassertion AS vopa
         ON vopa.object = opa.subject
       LEFT OUTER JOIN objectpropertyassertion AS nopa
         ON nopa.subject = opa.object
\end{lstlisting}

Diese Abfrage erzeugt ein Ergebnis in der alle objectPropertyAssertions einer Liste mit ihrem Vorgänger und Nachfolger aufgelistet sind. Außerdem wird die Anzahl der Elemente in der Liste mitgeführt.

\begin{lstlisting}[language=SQL]
INSERT INTO objectPropertyAssertion
           (subject,
            property,
            object)
SELECT start.start,
       pc.property,
       ende.ende
FROM   (SELECT   lname,
                 anz
        FROM     (--- Unterabfrage fuer gueltige Liste ---)
        GROUP BY lname
        HAVING   COUNT(lname) = anz) AS thel
       INNER JOIN (SELECT lname,
                          start
                   FROM   (--- Unterabfrage fuer Vorgaenger ---)
                   WHERE  vorgaenger IS NULL) AS start
         ON start.lname = thel.lname
       INNER JOIN (SELECT lname,
                          ende
                   FROM   (--- Unterabfrage fuer Nachfolger ---)
                   WHERE  nachfolger IS NULL) AS ende
         ON ende.lname = thel.lname
       INNER JOIN propertyChain AS pc
         ON pc.list = thel.lname
\end{lstlisting}

Die vorher beschriebene Abfrage wird mehrmals eingesetzt. Sie erzeugt einerseits nur gültige Liste. Das ist die erste Abfrage. Die zweite Unterabfrage wählt das Startelement aus und die dritte das Endeelement. Am Ende wird es noch auf die eigentliche propertyChain gejoined.

Die zweite Regel überprüft, ob ein Invidivum in allen Teilen einer Schnittmenge vorhanden ist. Falls ja wird es der Klasse dieser Schnittmenge hinzugefügt.
\begin{table}[htb]
\begin{center}
	\begin{tabular}{m{5cm}|m{3.5cm}}
	if & then \\ \hline
	T(?c, owl:intersectionOf, ?x),\newline
	LIST[?x, ?$c_1$, ..., ?$c_n$],\newline
	T(?y, rdf:type, ?$c_1$),\newline
	T(?y, rdf:type, ?$c_2$),\newline
	\ldots,\newline
	T(?y, rdf:type, ?$c_n$) & T(?y, rdf:type, ?c) 	 
	\end{tabular}
\end{center}
	\caption{Der Spezialfall cls-int2}
	\label{rule-cls-int2}
\end{table}


\begin{lstlisting}[language=SQL]
INSERT INTO classAssertionEnt
           (entity,
            class)
SELECT   clsA.entity AS entity,
         int.class   AS class
FROM     (SELECT   name,
                   COUNT(name) AS anzahl
          FROM     list
          GROUP BY name) AS anzl
         INNER JOIN list AS l
           ON anzl.name = l.name
         INNER JOIN classAssertionEnt AS clsA
           ON l.element = clsA.class
         INNER JOIN intersectionOf AS int
           ON int.list = l.name
GROUP BY l.name,
         clsA.entity,
         int.class
HAVING   COUNT(l.name) = anzl.anzahl
\end{lstlisting}

Diese Abfrage wurde auch wesentlich vereinfacht. Für die Abspeicherung einer Historie dürfen die Fakten erst danach gruppiert werden, dann wäre allerdings keine Überprüfung mit HAVING möglich.

Die aktuelle Umsetzung kann im Wiki oder im Code gefunden werden.

Die dritte Abfrage prp-key überprüft ob zwei Individuen das Gleiche sind. Dies geschieht an Hand einer Liste  von Key-Eigenschaften, wenn diese übereinstimmen sind auch die die Individuen gleich.
\begin{table}[htb]
\begin{center}
	\begin{tabular}{m{4.5cm}|m{4cm}}
	If & then \\ \hline
	T(?c, owl:hasKey, ?u),\newline
	LIST[?u, ?$p_1$, \ldots, ?$p_n$],\newline
	T(?x, rdf:type, ?c),\newline
	T(?x, ?$p_1$, ?$z_1$),\newline
	\ldots,\newline
	T(?x, ?$p_n$, ?$z_n$),\newline
	T(?y, rdf:type, ?c),\newline
	T(?y, ?$p_1$, ?$z_1$),\newline
	\ldots,\newline
	T(?y, ?$p_n$, ?$z_n$) & T(?x, owl:sameAs, ?y)
	\end{tabular}
\end{center}
	\caption{Der Spezialfall prp-key}
	\label{rule-prp-key}
\end{table}


Die Abfrage ist der Einfachheit ebenfalls in zwei Teile zerlegt. Die Unterabfrage, die mehrmals verwendet wird, erzeugt zunächst eine gültige Liste von Individuen die über eine Eigenschaft mit dem gleichen Objekt verbunden sind.
\begin{lstlisting}[language=SQL]
SELECT pax.subject, sl.name, anzl.anz
FROM list AS sl
        INNER JOIN (
                SELECT name, COUNT(name) AS anz
                FROM list
                GROUP BY name
        ) AS anzl
                ON anzl.name = sl.name
        INNER JOIN (
                SELECT id, subject, property, object
                FROM objectPropertyAssertion
                UNION
                SELECT id, subject, property, object
                FROM dataPropertyAssertion
        ) AS pax
                ON sl.element = pax.property
        INNER JOIN (
                SELECT id, subject, property, object
                FROM objectPropertyAssertion
                UNION
                SELECT id, subject, property, object
                FROM dataPropertyAssertion
        ) AS pay
                ON sl.element = pay.property
                AND pax.property = pay.property
                AND pax.object = pay.object
GROUP BY pax.subject, sl.name
--- Es werden alle Listenelemente doppelt erzeugt
HAVING COUNT(sl.name) = 2*anz
\end{lstlisting}

Diese Liste wird dann in der folgenden Abfrage zweimal verwendet um zwei Individuen miteinander verleichen zu können.
\begin{lstlisting}[language=SQL]
INSERT INTO sameAsEnt (left, right)
SELECT ca1.entity AS left, ca2.entity AS right
FROM hasKey AS hk
     INNER JOIN list AS l
             ON l.name = hk.list
     INNER JOIN classAssertionEnt AS ca1
            ON ca1.class = hk.class
     INNER JOIN classAssertionEnt AS ca2
            ON ca2.class = hk.class
     INNER JOIN (--- ...Unterabfrage... ---) AS valid1
                ON valid1.subject = ca1.entity
     INNER JOIN (--- ...Unterabfrage... ---) AS valid2
                ON valid2.subject = ca2.entity
\end{lstlisting}

\section{Delta-Iteration}
\label{abschnitt-delta-iteration}

Unabhängig von der Art der delta-Iteration muss immer der Ausführungskontext bekannt sein, d.h. welche delta-Iteration ist aktiv, auf welches delta wird gearbeitet. Ein delta muss dabei die Information enthalten auf welcher Relation es arbeitet und auf welchen Teil. Zum Ausführungskontext gehört je nach Modus auch noch die Information, ob eine Regel schon auf das Ziel delta angewendet wurde.

Die delta-Iteration ist ein Verfahren, um bei Regelanwendungen, die Datenmenge so zu reduzieren, das nur solche enthalten sind, die noch nicht berücksichtigt wurden und somit die tatsächlich zu betrachtenden Menge zu reduzieren.

Während der ersten Entwicklung sind dabei zwei Varianten der delta-Iteration klar geworden. Diese tragen im folgenden die Namen \emph{immediate} delta-Iteration und \emph{collective} delta-Iteration. Timo Weithöhner hat in seiner Implementierung die collective delta-Iteration verwendet.

\subsection{Immediate delta-Iteration}

Hierbei erzeugt jede Regelanwendung sofort ihr eigenes Delta, mit dem weitergearbeitet werden kann. Wenn Zeilen erzeugt wurden, wird sofort das dazugehörige Delta erzeugt und dafür alle notwendigen Regelanwendungen ausgelöst.

Im folgenden ist ein Beispiel gegeben, wie der Ablauf bei der Anwendung einer Regel ist die neue Fakten erzeugt hat und wie diese dann verarbeitet wird. Angenommen die Regel \emph{eq-ref-s} erzeugt neue Fakten, diese landen in einem Delta der \emph{sameAs}-Relation. Das löst eine ``Reason'' aus mit der Information welches Delta neue Fakten enthalten hat. Wenn der Regelprozessor diese erhält erzeugt er für diese Kombination sog. RuleActions, d.h. er bringt alle Regeln die auf diesem delta arbeiten in eine Warteschlange. Zu jeder Regel speichert er den Ausführungskontext, auf den sie angewendet werden soll mit ab. Die Regelanwendungen warten dann bis sie zur Ausführung gebracht werden.

\begin{verbatim}
Regel: eq-ref-s //neue Fakten
=> delta sameas
   => new Reason(sameas, delta)
      => new RuleAction(eq-sym, delta)
         new RuleAction(eq-trans, delta)
         new RuleAction(eq-rep-s, delta)
         new RuleAction(eq-rep-p, delta)
         new RuleAction(eq-rep-o, delta)
         new RuleAction(eq-diff1, delta)
         new RuleAction(eq-diff2, delta)
         new RuleAction(eq-diff3, delta)
\end{verbatim}

Der Regelprozessor ist dann auch verantwortlich die Ausführung der Regelanwendungen zu starten. Das passiert in einer Schleife, die die Elemente der Warteschlange abarbeitet bis keine weiteren Elemente vorhanden sind.

Dies ist im folgenden Code \ref{code-immediate-delta-iteration} schematisch dargestellt:

\begin{figure}[htp]
	\caption{Abarbeitung der RuleActions im immediate Modus.}
	\label{code-immediate-delta-iteration}
	\begin{lstlisting}[language=Java]
while(action = rulesToApply.popAction()) {
	action.rule.apply(action.delta);
	
	if (!(rulesToApply.contains(delta)) {
		delta.addToRelation();
		delta.drop();
		delta == null;
	}
}
	\end{lstlisting}
\end{figure}


Der Vorteil dieser Vorgehensweise ist das es besser parallelisiert werden kann, da so viele Regelanwendungen wie möglich gleichzeitig angewandt werden können. Die Parallelisierung ist eines der Ziele die eventuell in der Zukunft umgesetzt werden sollen.

Der Nachteil ist das viele Deltas gleichzeitig entstehen und insgesamt mehr Regel angewendet werden müssen. Die Deltas enthalten unter Umständen nur wenig Inhalt, d.h. die Abfragen arbeiten auf kleineren Datenmenge aber ein Datenbanksystem lohnt sich eventuell erst wirklich bei großen Datenmengen. Es muss nicht zwangsläufig schlecht sein, aber es ist nicht unbedingt klar wie sich das auf ein DBMS auswirkt.

\subsection{Collective delta-Iteration}

Vom aktuellen Stand einer Relation werden erst einmal alle Änderungen in einer Hilfsrelation gesammelt. Sind alle abgearbeitet, wird das Delta erstellt und die nächste Phase beginnt.

\begin{verbatim}
Regel: sameas-eq-ref-s //neue Fakten
=> aux sameas
\end{verbatim}

\begin{figure}[htp]
	\caption{Abarbeitung der RuleActions im immediate Modus.}
	\label{code-immediate-delta-iteration}
	\begin{lstlisting}[language=Java]
do {
	while (action = list.popAction()) {
		action.rule.apply(action.delta);
	}
}
while (applyRelations()); //WAIT --- SYNC

boolean applyRelations() {
	for(r : relations)
		r.applyAux();
		return relations.wereDirty()
}

applyAux() {
	createDelta;
	addDelta();
	clearAux();
	new Reason(relation, delta);
}
	\end{lstlisting}
\end{figure}

Vorteile:
\begin{itemize}
  \item Es gibt immer nur eine Hilfsrelation und ein Delta, also eine fixe Anzahl an Tabellen.
  \item Es werden evtl. größere Menge an neuen Fakten zusammengefasst. Das muss nicht notwendiger weise gut sein.
\end{itemize}

Ein Nachteil ist, das die Parallelisierung nicht vollständig durchgeführt werden kann, da zu gewissen Zeitpunkten (Zeile 6), die Regelanwendung unterbrochen wird und die Ausführung wieder synchronisiert werden muss.

\subsection{Simulation}

Die collective delta-Iteration kann mit Hilfe der immediate delta-Iteration und ein paar Änderungen umgesetzt werden. Dazu dürfen die Relationen nicht sofort neue Hinweise an den Regelprozessor schicken, sondern es gibt eine eigene Phase nach dem Abarbeiten aller Regeln. Dabei wird erst der Inhalt der Deltas in die Haupttabelle übernommen und es werden die Hinweise an den Regelprozessor geschickt.

Die Hilfstabelle in der die Daten für die Runde zwischengespeichert werden ist das Delta, das neu angelegt wird. Die alte Hilfstabelle wird zum Delta der aktuellen Runde.

Die Implementierung des u2r3 unterstützt beide Varianten der delta-Iteration. Diese können in der Konfigurationsdatei mit der Option DeltaIteration verändert werden. Gültige Werte sind \emph{COLLECTIVE} oder \emph{IMMEDIATE}.



\section{Abfragen}
\subsection{Komplexe Ausdrücke finden}
Beim Suchen von komplexen Ausdrücken hat man es damit zu tun, das der Ausdruck zur Speicherung in den Relationen aufgeteilt werden musste. Diese Aufteilung ist mit von sog. NodeId passiert, wie sie in der OWLAPI auch für anonyme Individuen verwendet wird, werden sie auch hier verwendet, um eine Beziehung zwischen Teilen herzustellen. Was aber der Wert dieser IDs war ist bei einer späteren Anfrage nicht mehr bekannt bzw. es ist auch nicht bekannt wie diese auf den komplexen Ausdruck passen sollten oder ob der Ausdruck überhaupt in der geschlussfolgerten Menge liegt.

Um den Ausdruck nicht in die Datenbank einfügen zu müssen und ihn durch Schlussfolgern gleichsetzen zu können wird dieser Ausdruck rekursiv durchlaufen und daraus ein SQL-Ausrduck aufgebaut. Dieser SQL-Ausdruck kann den Namen der Relation, sowie die ID des Konstrukts zurückliefern. Beim Aufbau der SQL-Abfrage erzeugt jedes Unterkonstrukt des komplexen Ausdrucks eine SQL-Unterabfrage. Dabei ist zu beachten, das die Unterabfragen eine Referenz zu ihrer ``Oberabfrage'' enthalten.

Nachfolgend ein Beispiel, dabei wird ein komplexer OWL-Ausdruck\footnote{Der Ausdruck wird in der FunctionalSyntax dargestellt. Damit ist die Hierarchie so wie das rekursive Vorgehen leichter nachvollziehbar.}

\begin{verbatim}
ClassAssertion(ind1, SomeValuesFrom(prop1, cls1))
\end{verbatim}

in eine SQL-Abfrage umgewandelt.

\begin{lstlisting}[language=SQL]
SELECT id, 'classAssertion'
FROM classAssertion AS t1
WHERE entity = 'ind1' AND
	EXISTS (
	SELECT id, 'someValuesFrom'
	FROM someValuesFrom AS t2
	WHERE t2.class = t1.class AND
		property = 'prop1' AND
		total = 'cls1'
	)
\end{lstlisting}

Beim Löschen werden ebenfalls möglicherweise komplexe Ausdrücke angegeben. Um diese aufzufinden können die selben Methoden verwendet werden.

\section{Löschung (\& Änderung)}

Wie aus dem Grundlagen Kapitel \ref{kapitel-grundlagen} bekannt ist handelt es sich bei OWL2 RL um eine monotone Logik. Damit ist das Hinzufügen von Fakten kein Problem, da es sich nicht auf schon geschlussfolgerte Fakten auswirkt. In der Implementierung ist damit die Änderung von Fakten damit umgesetzt, das die zu ändernden Fakten gelöscht und dann die neuen Fakten wieder hinzugefügt werden. Die Änderung von Fakten ist damit auf die Löschung von Fakten reduzierbar.

Allgemein gibt es zwei Arten wie die Löschung von Fakten umgesetzt werden kann. Der naive Ansatz wär es, wenn ein Faktum gelöscht werden soll, werden alle bisher geschlussfolgerten Ergebnisse verworfen und der Schlussfolgerungsprozess wird danach von vorne gestartet. Ein optimierter Ansatz wäre es, das ein Faktum das gelöscht wird nur die Löschung von daraus geschlussfolgerten Ergebnissen auslöst. Dieses Vorgehen wurde im Schlussfolgerer umgesetzt und im folgeden näher beschrieben. Zunächst die grundlegenden Bedingungen für das Löschen von Fakten:

\begin{itemize}
  \item Fakten die hergeleitet wurden können nicht gelöscht werden.
  \item Fakten die keine Ableitungen verursacht haben können gelöscht werden.
  \item Fakten die eine Ableitung verursacht haben können nur gelöscht werden, wenn alle ihre Ableitungen gelöscht werden.
\end{itemize}

Um zu ermöglichen, dass das Löschen eines Faktums nur Ergebnisse entfernt die daraus abgeleitet wurden wird die Ableitungsreihenfolge abspeichern mit abgespeichert. Durch diesen zusätzlichen Aufwand soll die Manipulation an der Ontologie beschleunigt werden.

Dabei ist die Frage wichtig, woher etwas abgeleitet ist. Regeln erzeugen diese Fakten, aber damit eine Regel ausgelöst wird sind eigentlich die Fakten verantwortlich. Das Problem hierbei ist, das unterschiedliche Regeln eine unterschiedliche Anzahl an Fakten in der Prämisse verwenden. Die drei Spezialregeln sind dabei besonders schlimm da sie eine theoretisch beliebig große Anzahl Fakten in der Prämisse verwendet.

Gut ist hingegen das durch das MEMA Prinzip die Unterscheidung zwischen T-Box und A-Box aufgehoben wird. Der Regelsatz von OWL2 RL nimmt darauf auch keinen besonderen Bezug. Damit ist möglich für Löschungen in der T-Box und in der A-Box das selbe Verfahren verwenden zu können.

\subsection{Abspeichern der Ableitungsreihenfolge}
Die ApplicationRules sind die Klasse aller Regeln die neue Fakten erzeugen können. Diese wurden so erweitern, dass sie von jeder verwendeten Tabelle in der Regel, die id der Zeile mitliefert. Jede Tabelle wurde so ausgestatte das jede Zeile als jeder Fakt eine eindeutige Nummer hat. So kann jedes Konstrukt einer Tabelle über eine Nummer angesprochen werden, egal wieviele Unterelemente dieses Konstrukt hat. Das entspricht sehr dem MEMA-Schema, da durch eine id eine klare, einheitlich Historien-Tabelle angelegt werden kann, die unabhängig von unterschiedlichen Axiom-Konstrukten ist. Die ids müssen einzigartig pro Tabelle sein.

Die Anzahl der Spalten die eine Regelanwendung für die Angabe der Historiendaten ist von Regel zu Regel unterschiedlich aber dabei immer für jede Regel fest. Davon ausgenommen sind die drei Spezialregeln. Wie die Historiendaten hier mitgeschrieben werden, wird in Abschnitt \ref{abschnitt-aufblaehung} geklärt.

Danach müssen die Historiendaten in eine seperate Referenztabelle (history \ref{relations-list-history}). Hier wird zu einer id und einer Tabelle, die Quelldaten woher dieser Fakt stammt ebenfalls in Form von einer id und Tabelle abgepseichert.

Im nachfolgendes Beispiel wird die Situation angenommen das eine Subklassenbeziehung zwischen A,B, und B,C besteht. Damit besteht auch eine Subklassenbeziehung zwischen A,C. All diese Ausdrücke sind mit einer eindeutigen id identifizierbar.
\begin{verbatim}
subClass(A,C) := subClass(A,B), subClass(B,C)
   id_neu             id_x           id_y
\end{verbatim}

Dadurch werden folgende Zeilen in der Historietabelle angelegt.

\begin{table}[htb]
\begin{center}
\begin{tabular}{l|l}
ID & QuellID \\ \hline
$id_{neu}$ & $id_x$ \\
$id_{neu}$ & $id_y$
\end{tabular}
\end{center}
\caption{Anlegen einer Ableitungshistorie}
\label{table-inference-history}
\end{table}
Um in deltas schon eine deterministische id vergeben zu können wird eine tabellenübergreifende Sequence verwendet. Das ist eine Möglichkeit in einer Datenbank Standardwerte für Spalten aus einer zentralen Stelle vergeben zu können.

\subsection{Löschablauf}
Man beginnt das Löschen eines Axioms damit das man seine id bestimmt. Dann überprüft man, ob die id in der Historientabelle vorhanden ist. Wenn ja, dann alle Abhängigkeiten finden und rekursiv diese löschen. Danach den Wert selbst löschen. 

z.B.: $subClass(B,C) \Rightarrow id_y$
\begin{lstlisting}[language=SQL]
SELECT id FROM Historie WHERE quelle = $id_y$

while(id)
	delete(id)
	DELETE FROM Historie WHERE id = id

DELETE FROM subClass WHERE id = $id_y$
\end{lstlisting}


Nachdem Fakten gelöscht wurde müssen ebenfalls Regeln ausgelöst werden, da es sein könnte das gelöschte geschlussfolgerte Ergebnisse auch über andere Fakten ableitbar gewesen wären. 
Hier wär auch ein naiver Ansatz möglich, indem man alle Regeln auf alle bekannt Fakten anwendet. Es wurde aber ein effizienteres Verfahren implementiert, so dass nur Regeln ausgelöst werden, die Fakten in den Relationen erzeugen können aus denen etwas gelöscht wurde.
Um entscheiden zu können welche Regeln ausgelöst werden sollen, müssen die Relationen wissen welche Regeln in ihnen neue Fakten erzeugen können. Diese Regeln müssen dann ausgelöst werden im Gegenteil zu ''normalen'' Inferenzregeln.

Für die Unterscheidung wurden zwei verschiedene Reason eingeführt werden.
\begin{itemize}
  \item AdditionReason (wenn der Grund durch das Hinzufügen eines Faktes entstanden ist)
  \item DeletionReason (wenn der Grund durch das Löschen eines Faktes entstanden ist)
\end{itemize}

Der Modus mit dem gelöscht wird kann in der Konfigurationsdatei von u2r3 eingestellt werden. Die Option dafür hat den Namen \emph{DeletionType} und hat die gültigen Werte \emph{CLEAN} oder \emph{CASCADING}.

\subsection{Anmerkung zu Aufblähung von Deltas}
\label{abschnitt-aufblaehung}
Welche Regeln sind das:
Wird nur von den drei Sonderregeln verursacht. Wenn sie nichts erzeugen blähen sie auch nichts auf, wenn doch.

Lösungsmöglichkeit. Nach dem bearbeiten dieser Regeln, sofort die Regelqueue flushen, mergen und nochmal anfangen.

Spielt also erstens nur eine Rolle wenn im COLLECTIVE Mode.

Nach dem mergen ist das Delta aber sowie so eingeschrumpft. Auf das Schlussfolgern wirkt es sich also kaum negativ aus.

\section{Konsistenzprüfungen}
Um die Konsistenz einer Ontologie aus dem OWL2 RL Profil sicherzustellen, sind in der Spezifikation eigene Regeln vorgegeben. Es ist dabei aber nicht vorgeschrieben wann diese Regeln zur Überprüfung zur Ausführung gebracht werden müssen. Außerdem muss eine Ontologie unter Umständen nicht immer auf ihre Konsistenz überprüft werden, daher ergeben sich zunächst folgende Ideen wann und wie man die Prüfungen durchführt.

\begin{itemize}
  \item Nach jeder Regelanwendung: Dies ist nicht sinnvoll, da durch eine Regelanwendung nicht gegen beliebige Konsistenzregeln verstoßen werden kann.
  \item Nur abhängige Konsistenzregeln überprüfen: Die Auslösung der Regeln funktioniert dabei genauso wie bei gewöhnlichen Regeln. Dieses Verfahren meldet sofort zurück wenn eine Inkonsistenz gefunden wurde, macht aber nur den nötigsten Aufwand. Es ist in u2r3 implementiert und ist der voreingestellt Modus zur Konsistenzprüfung.
  \item Die Überprüfung könnte erst nach kompletter Ableitung durchgeführt werden. Eine Inkonsistenz wird dann möglicherweise erst spät festgestellt.
  \item Keine Überprüfung der Konsistenz ist sicherlich die schnellste Möglichkeit, da hierbei einfach keine Regeln ausgeführt werden müssen. Dies ist hilfreich, wenn man z.B. die Konsistenz schon in einem vorherigen Lauf bestimmt hat. Diese Option ist ebenfalls implementiert.
\end{itemize} 
Die Möglichkeit die Konsistenzprüfung zu wechseln ist in u2r3 implementiert und kann in der Konfigurationsdatei unter dem Namen \emph{ConsistencyLevel} gesetzt werden. Gültige Werte dabei sind \emph{NONE} oder \emph{DEFAULT}.

Damit der Regelprozessor zwischen Konsistenzregeln und Anwendungsregeln unterscheiden kann sind alle Regeln, die eine Inkonsistenz auslösen können von der Unterklasse ConsistencyRule abgeleitet.

\subsection{Rückmeldung}
Konsistenzregeln melden sich nur in einem Fehlerfall zurück. Die Rückmeldung erfolgt aus den Regeln selbst während ihrer Ausführung. Die OWLAPI schlägt hier vor eine Exception zu werfen. Allerdings sind nicht alle Inkonsistenz gleich schwerwiegend. Eine Inkonsistenz in der A-Box ist normalerweise nicht so problematisch wie eine Inkonsistenz in der T-Box.

Die Möglichkeiten wie auf eine Inkonsistenz reagiert werden kann sind dabei:
\begin{itemize}
  \item Nur eine Warnung auszugeben, diese wird dann automatisch mitgeloggt. Das ist die Voreinstellung. Damit wird der Schlussfolgerungsprozess nie unterbrochen.
  \item Abhängig von der Inkonsistenz reagieren. Dabei wird zwischen Inkonsistenzen in der A-Box und der T-Box unterschieden. Ein Fehler in der A-Box gibt nur eine Warnung aus, wohingegen ein Fehler in der T-Box eine Ausnahme erzeugt.
  \item Die Verarbeitung unterbrechen und bei jeder Inkonsistenz eine Exception erzeugen.
\end{itemize}

Im Reasoner kann dies mit der Option \emph{InconsistencyReaction} verändert werden, dabei sind \emph{WARN}, \emph{PERCASE} oder \emph{FAIL}.



