\chapter{Ausblick}
\label{kapitel-ausblick}

\section{Weitere Optimierungen}
Der Schlussfolgerer wurde immer mit dem  Hintergedanken entwickelt Parallelisierung und Approximation beim ableiten zu ermöglichen. Im Laufe der Arbeit wurden aber noch andere Möglichkeiten zur Verbesserung fedunden, die allerdings auf Grund von Zeitmangel nicht oder nur unvollständig umgesetzt werden konnten.

\begin{itemize}
  \item Die SQL-Abfragen weiter optimieren. Vor allem im nicht kaskadierenden Modus können einige Abfragen mit Datenbank spezifischeren Konstrukten ersetzt werden. Das erzeugt aber mehr Fallunterscheidungen und muss einzeln geprüft werden, ob die Abweichung von der Standardumsetzung korrekt ist.
  \item Man könnte manche Regeln, insbesondere dt-eq mit anderen zusammenlegen, da wenn sie alleine laufen gelassen wird recht aufwendig ist. Hierbei könnte man sich auch gleich über einen optimaleren Satz an Regeln für das OWL2 RL Fragment Gedanken machen.
  \item Der Regelsatz könnte noch erweitert werden, wie das Beispiel in der DomusAG-Ontologie zeigt. Vorallem im Bereich der Datatype Restriction kann man noch einiges schlussfolgern.
  \item Es wäre vorstellbar eine Lösung zwischen der Abspeicherung der vollen Ableitungshierarchie und keiner Abspeicherung zu finden. Besonders bei den drei Spezialregeln ist der Aufwand hoch die Ableitungshierarchie mit abzuspeichern. Man könnte hier eine Art Wildcard benutzen. Das sähe dann folgendermaßen aus, das alle Fakten, die schon von einem Fakt mit einer Wildcard abgeleitet wurden wieder eine Wildcard bekommen. Wird dann eine Fakt gelöscht wird es und alle seine Abhängigkeiten plus alle Fakten die eine Wildcard haben gelöscht.
  \item Die Zahl der Regeln, die angewendet werden könnte man noch weiter reduzieren. Relationen wissen, ob sie Fakten enthalten oder nicht. Wenn eine Rgel ausgelöst werden soll die auf einer Relation arbeitet die keine Fakten enthält könnte sie schon vorher abgebrochen werden.
  \item Im Moment sind die Literal über verschiedene Relationen verteilt. Würde man diese in einer zentralen Relation halten würden sich viele Rgeln vereinfachen und die Daten müssten nicht hin und her kopiert werden.
  \item Die Reihenfolge der Regelanwendung könnte man optimieren. Hier könnte man versuchen die Regeln einer Reihenfolge anzuwenden, um auf die Erzeugung von gewissen Fakten hinzuarbeiten oder man könnte versuchen Regeln so anzuwenden, das sie möglichst wenig neue Regeln auslösen bzw. ``leichtere'' Regeln bevorzugen.
  \item Beim Hinzufügen von neuen Axiomen könnt man diese gleich in eigene Deltas schreiben. Somit würde man das erneute Schlussfoglern beschleunigen, da nur mit den neuen Fakten gearbeitet werden muss.
  \item In den Ontologien werden IRIs zur Identifizierung verwendet. Man könnte hier den Speicherverbrauch reduzieren, in dem man die IRIs in einer gesonderten Relation ablegt, sie dort mit Ganzzahlen kodiert und ansonsten nur mit diesen Zahlen arbeitet.
\end{itemize}

\section{Zukunft}
OWL2 RL soll in seiner Ausdrucksmächtigkeit in kommenden Versionen noch vergrößert werden. Vielleicht sind damit Facets möglich.

OWL2 wurde erst im Verlaufe dieser Diplomarbeit als Recommendation veröffentlicht, aber die Weiterentwicklung von OWL wird schon angedacht \cite{Hitzler2009} und es werden verschieden Konferenzen zu diesem Thema abgehalten. Auf \url{http://semanticweb.org/wiki/Events}
findet man eine große Anzahl an Konferenzen und Workshops die zu diesem Gebiet gehören.

\section{Probleme und Chancen}
Im Moment nimmt die Anzahl an Ontologien drastisch zu. Das OWL2 RL Profil ist allerdings noch neu und es sind kaum Tools für einen produktiven Einsatz vorhanden. Ein Großteil der Anwendungen hat eine große Überdeckung mit dem Zielgebiet von OWL2 RL. Die Frage ist allerdings dabei werden die neu enstehenden Ontologien sich an dieses Profil halten. Wie gut kann man alte konvertieren und wie schnell werden vernünftige Werkzeuge zu Verfügung stehen. Die Zeit ist eigentlich reif für OWL2 RL, fraglich ist nur ob OWL2 RL und sein Umfeld die Anforderungen erfüllen kann.

Die Einsatzgebiete sind schon aus der Einführung bekannt es kommen aber ständig neue hinzu. Möglichen Einsatzgebieten sind quasi keine Grenzen gesetzt. Es zeichnet sich aber ein Trend für Desktop-Anwendungen und für den Web-Bereich ab.