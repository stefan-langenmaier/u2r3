\chapter{Ausblick}
\label{kapitel-ausblick}

\section{weitere Optimierungen}
Speicherreduzierung

Nicht die URIs abspeichern sondern nur Referenznummer in eine seperate Tabelle. Evtl funktionieren dann auch die Abfragen schneller (geringeres Datenaufkommen, effizientere Typen)

\begin{verbatim} 
* Kann man quadrat. Rekursion immmer auflösen?
* Kann man manche Ableitungsschritte zusammenfassen?
* Was bringen rekursive Abfragen wirklich? (Es werden dadurch ja auch andere Regeln angestoßen und dann müssen die rekursiven wieder ran)

==== Idee ====
Evtl. können auch Daten mit einer Art Wildcard angelegt werden. Daten, die keine Quellangabe haben werden immer gelöscht.

\end{verbatim}

\section{Zukunft}
OWL2 RL soll in seiner Ausdrucksmächtigkeit in kommenden Versionen noch vergrößert werden. Vielleicht sind damit Facets möglich. [Hitzler2009, Suggestions for OWL 3]

\subsection{Probleme und Chancen}
Im Moment nimmt die Anzahl an Ontologien drastisch zu. Das OWL2 RL Profil ist allerdings noch neu und es sind kaum Tools für einen produktiven Einsatz vorhanden. Ein Großteil der Anwendungen hat eine große Überdeckung mit dem Zielgebiet von OWL2 RL. Die Frage ist allerdings dabei werden die neu enstehenden Ontologien sich an dieses Profil halten. Wie gut kann man alte konvertieren und wie schnell werden vernünftige Werkzeuge zu Verfügung stehen. Die Zeit ist eigentlich reif für OWL2 RL, fraglich ist nur ob OWL2 RL und sein Umfeld die Anforderungen erfüllen kann.

Die Einsatzgebiete sind schon aus der Einführung bekannt es kommen aber ständig neue hinzu. Möglichen Einsatzgebieten sind quasi keine Grenzen gesetzt. Es zeichnet sich aber ein Trend für Desktop-Anwendungen und für den Web-Bereich ab.